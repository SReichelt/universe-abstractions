\documentclass[a4paper]{article}

\usepackage[margin=2.5cm]{geometry}
\usepackage[parfill]{parskip}
\usepackage[utf8]{inputenc}
\usepackage{hyperref}

\usepackage{amsmath,amssymb,amsfonts,amsthm}
\usepackage{longtable}

\title{Universe Abstractions\footnote{Still searching for a better title.}}
\author{Sebastian Reichelt}

\setlength{\skip\footins}{0.5cm}

\theoremstyle{definition}
\newtheorem{definition}{Definition}[section]
\newtheorem{theorem}[definition]{Theorem}
\newtheorem{proposition}[definition]{Proposition}
\newtheorem{lemma}[definition]{Lemma}
\newtheorem{corollary}[definition]{Corollary}
\newtheorem{conjecture}[definition]{Conjecture}
\theoremstyle{remark}
\newtheorem*{remark}{Remark}
\newtheorem*{remarks}{Remarks}
\newtheorem*{example}{Example}

\newcommand{\defn}{\emph}

\renewcommand{\phi}{\varphi}
\renewcommand{\equiv}{\simeq}
\renewcommand{\emptyset}{\varnothing}

\newcommand{\A}{\mathcal{A}}
\newcommand{\B}{\mathcal{B}}
\newcommand{\C}{\mathcal{C}}
\newcommand{\D}{\mathcal{D}}
\newcommand{\U}{\mathcal{U}}
\newcommand{\V}{\mathcal{V}}
\newcommand{\W}{\mathcal{W}}

\newcommand{\nm}{\mathsf}

\newcommand{\universe}{\nm}
\newcommand{\Unit}{\universe{Unit}}
\newcommand{\Bool}{\universe{Bool}}
\newcommand{\Prop}{\universe{Prop}}
\newcommand{\Set}{\universe{Set}}

\newcommand{\id}{\nm{id}}
\newcommand{\map}{\nm{map}}
\newcommand{\congrArg}{\nm{wd}}
\newcommand{\fndef}[1]{\nm{def}_{#1}}

\newcommand{\combinator}{\nm}
\newcommand{\idFun}{\combinator{I}}
\newcommand{\appFun}{\combinator{T}}
\newcommand{\constFun}{\combinator{K}}
\newcommand{\compFun}{\combinator{B'}}
\newcommand{\revCompFun}{\combinator{B}}
\newcommand{\swapFun}{\combinator{C}}
\newcommand{\dupFun}{\combinator{W}}
\newcommand{\substFun}{\combinator{S'}}
\newcommand{\revSubstFun}{\combinator{S}}

\newcommand{\Lean}{\texttt}

\begin{document}

\maketitle

\section{Introduction}

In this document, we define a mathematical framework that `automates' many proofs of structural properties,
both in a practical sense and as an abstract theory: On the one hand, the accompanying formalization in the
Lean 4 theorem prover\cite{Lean} can be used to derive lemmas about a wide range of mathematical objects,
starting from relatively few axioms that these objects must satisfy. At the moment, this writeup is
mainly intended as documentation for the Lean formalization. On the other hand, as the mathematical content
may be interesting and useful in its own right, here we describe it in a conventional fashion without any
Lean-specific prerequisites.

The framework is influenced by ideas from Homotopy Type Theory\cite{HoTT}, but rather than offering
a new foundation of mathematics, we transfer those ideas to more traditional mathematical settings: In this
document, we work in Zermelo-Fraenkel (ZF) set theory with universes, whereas the Lean formalization is
directly based on Lean's foundation without any additional axioms. This shows that our theory is as
`foundation-agnostic' as most other mathematical theories, which has the advantage that it can be used
side-by-side with any other theory instead of having to fit all existing mathematics into it.
(It can also be formalized \emph{within} Homotopy Type Theory, but we will not cover this.)

The initial insight, on which the rest of the framework is built, is that categories and their functors
form a simply-typed lambda calculus in which the $\nm{S}$ and $\nm{K}$ combinators exist (theorem
\ref{thm:funop}).
Therefore, functors can be derived algorithmically from lambda terms, and it is possible to extend this
algorithm to incorporate other generic structures. The end result is that functions satisfying certain
conditions are automatically functorial.

Similarly, it is possible to automatically derive a proof that two functors are naturally isomorphic
from a proof that both functors map a fixed argument to equivalent values (theorem
\ref{thm:extensionality}).

The main goal is to automate proofs of isomorphism invariance in a similar way, but this is till work in
progress.

% Before we begin describing the theory formally, we would like to briefly explain what kind of `automation'
% our theory offers -- actually two distinct but related kinds of automation:
% \begin{itemize}
%   \item Although we do not assume any knowledge of Homotopy Type Theory (HoTT), we expect
%   many readers to be aware of its univalence axiom, which roughly states that isomorphic structures
%   can be assumed to be equal. Therefore, when working within HoTT, it is never necessary to
%   prove that a given property is isomorphism-invariant, as that is guaranteed by the constraints of
%   HoTT\footnote{Strictly speaking, the constraints are imposed by the variant of Martin-Löf dependent
%   type theory that HoTT is based on.}
%   that make the univalence axiom admissable. Similarly, our theory shows how certain
%   syntactic constraints generically guarantee e.g.\ isomorphism invariance and functoriality in a lot of
%   cases, but with the important difference that the constraints apply \emph{locally} to individual
%   mathematical definitions. (For a concrete example, see proposition \ref{prp:functoriality}.)
%   \item In the context of theorem provers, the word `automation' specifically means that the computer is
%   able to construct certain proofs automatically, and indeed the deduction of properties from syntactic
%   constraints that we just mentioned can be (and has partially been) automated in the Lean 4 theorem
%   prover.
% \end{itemize}

% \subsection{Example}
% 
% For an initial glimpse of the `automation' provided by the framework, we will briefly sketch how a simple
% algebraic data structure, in particular the class of all semigroups, fits into the framework, and what we
% can derive from that. Note that this example is somewhat `degenerate' compared to the much more generic
% framework: In particular, in the general case we avoid any use of equality.
% 
% For a set $S,$ let $\Phi(S)$ denote the set of all semigroup structures on $S,$ i.e. each element of $\Phi(S)$
% is an associative operation $\ast : S \times S \to S.$ The (class) function $\Phi$ is \defn{functorial}
% with respect to bijections between sets in the following sense: For any two sets $S,T$ we have a function
% $\phi : S \equiv T \to \Phi(S) \equiv \Phi(T)$, where $S \equiv T$ is the set of bijections between $S$
% and $T,$ satisfying
% \begin{itemize}
%   \item $\phi(\id_S) = \id_{\Phi(S)}$ for all sets $S,$ and
%   \item $\phi(g \circ f) = \phi(g) \circ \phi(f)$ for all sets $S,T,U$ and bijections $f : S \equiv T$ and
%   $g : T \equiv U.$
% \end{itemize}
% 
% Specifically for semigroups, these axioms are satisfied by
% $\phi(f)(\ast) := \star$ with $t_1 \star t_2 := f(f^{-1}(t_1) \ast f^{-1}(t_2))$ for $f : S \equiv T$ and
% $t_1,t_2 \in T.$ This is obviously equivalent to the standard condition that
% $f(s_1 \ast s_2) = f(s_1) \star f(s_2)$ for $s_1,s_2 \in S,$ but note how we did not specify this
% condition in advance but how it almost follows from the functoriality axioms on $\phi.$
% 
% In fact, the functoriality of $\Phi$ given by $\phi$ is sufficient to define the set of \defn{isomorphisms}
% between two semigroups $(S,\ast)$ and $(T,\star)$ generically as the set of bijections $f : S \equiv T$ for
% which $\phi(f)(\ast) = \star.$
% Readers familiar with Homotopy Type Theory may recognize that this definition matches equality of $\Sigma$
% types under the assumption of univalence\cite{_}, and that we can regard the condition
% $\phi(f)(\ast) = \star$ as a dependent equality $\ast =_f \star.$
% We will say more about this later.

We define the framework in such a way that categories are just one (important) special case, i.e.\ we define
axioms that are satisfied by categories but also by other structures that we define (and conjecturally also
by higher categories and certain topological spaces). Therefore, we start with some very generic definitions
that unify all of these structures.

\section{Universes}
\label{sec:universes}

As mentioned in the introduction, we work in Zermelo-Fraenkel (ZF) set theory with (a finite number of)
Grothendieck universes, which we will use liberally: Since we have formalized our theory in the Lean theorem
prover, and the foundation of Lean has been proved to be equiconsistent with ZF plus choice and
universes\cite{Lean-ZFC}, we relegate all universe-related details to the Lean formalization.

Therefore, we reserve the word `universe' for our own use except where specified otherwise:

\begin{definition}
  We define a \defn{universe} to be a pair $(I, (S_A)_{A \in I})$ of
  \begin{itemize}
    \item an index set $I,$ the members of which we call \defn{types}, and
    \item a family $(S_A)_{A \in I}$ of sets indexed by a type. We call the members of $S_A$ the
    \defn{instances} of $A.$
  \end{itemize}
\end{definition}

Given a universe $\U = (I, (S_A)_{A \in I}),$ we write
\begin{itemize}
  \item ``$A \in \U$'' for ``$A \in I$'' and
  \item ``$a :_\U A$'' or usually just ``$a : A$'' for ``$a \in S_A.$''
\end{itemize}

Despite the use of type-theoretic notation, we would like to stress that we are not defining a type
theory; our definitions are just \emph{informed} by type theory. Whenever we use the notation ``$a : A$,''
note that in general $A$ is not the set that $a$ is a member of; it can actually be an arbitrary object
from which that set can be obtained (similarly to Tarski-style universes in type theory\cite{Tarski}).

That makes the definition quite flexible. Fix a Grothendieck universe $U.$ Then we have the following
examples of universes according to the definition above.

\begin{itemize}
  \item For every collection $\C \subseteq U$ of sets in $U,$ $\Set_\C := (\C, (S)_{S \in \C})$ is a
  universe for which ``$:$'' and ``$\in$'' coincide.
  If $\C = U$, we just write ``$\Set$'' (again, ignoring all details about Grothendieck universes).
  \item For every collection $\C$ of structures of the form $(S, t_S)$ with $S \in U,$
  $(\C, (S)_{(S, t_S) \in \C})$ is a universe.\\
  As an example, let $\C$ be the set of all $U$-small groups. Then each type $A$ is a group, and
  each instance $a : A$ is member of the carrier set of $A.$\\
  The universe of categories will be of particular importance, where types are categories and instances
  are objects in those categories.\\
  Another example is that $\C$ is a collection of universes. Then each type is itself a universe, and
  its instances are the types in that universe.
  \item The previous example generalizes to structures that are built not on sets but on types in a
  universe $\U = (I, (S_A)_{A \in I}).$ Let $\C$ be a collection of structures of the form $(A,t_A)$ with
  $A \in \U$. Then $(\C, (S_A)_{(A, t_A) \in \C})$ is also a universe.
  \item For a universe $\U = (I, (S_A)_{A \in I}),$ any subset $J \subseteq I$ gives rise to a
  \defn{subuniverse} $(J, (S_A)_{A \in J}).$
  \item For any two universes $\U = (I, (S_A)_{A \in I})$ and $\V = (J, (T_B)_{B \in J}),$ we have a
  product universe $\U \times \V := (I{\times}J,\,(S_A{\times}T_B)_{(A,B) \in I{\times}J})$, as well as a
  sum universe $\U \uplus \V := (I \uplus J,\,(R_A)_{A \in I \uplus J})$ with $R_A := S_A$ for $A \in I$
  and $R_B := T_B$ for $B \in J$ (where $I \times J$ denotes the Cartesian product and $I \uplus J$
  denotes the disjoint union of $I$ and $J$).
\end{itemize}

Furthermore, we define two specific universes of interest:
\begin{itemize}
  \item $\Bool := \Set_{\{0, 1\}}$, where $0$ and $1$ are to be understood as von Neumann
  ordinals\cite{Ordinals}, so that $0$ is an empty type and $1$ is a type with a single
  instance $\emptyset : 1.$\footnote{In the Lean formalization of this theory, we have two different universes
  corresponding to the Lean types \Lean{Bool} and \Lean{Prop}. Both of them map to $\Bool$ in this document.}
  \item $\Unit := \Set_{\{1\}}.$
\end{itemize}

Universes are too generic to prove general statements about all universes, so in the following sections
we will define additional structure that universes may or may not have, and prove statements depending
on such additional structure.

\section{Meta-relations}
\label{sec:meta-relations}

\begin{definition}
  For a set $S$ and a universe $\V = (J, (T_B)_{B \in J}),$ we define a \defn{$\V$-valued meta-relation}
  on $S$ to be a function $(\prec) : S \times S \to J.$
\end{definition}

(We reserve the word ``relation'' for section \ref{sec:relations}, where we replace the set $S$ with a type
in a universe.)

We will write $(\prec)$ in infix form, but note that $(\prec)$ is a function and for $a,b \in S,$ the
expression $a \prec b$ is not a formula but a type in $\V.$
We say that $(\prec)$ is
\begin{itemize}
  \item \defn{reflexive} if for every $a \in S$ we have an instance $\id_a : a \prec a,$
  \item \defn{symmetric} if for every $a,b \in S$ and $f : a \prec b$ we have an instance
  $f^{-1} : b \prec a,$ and
  \item \defn{transitive} if for every $a,b,c \in S,$ $f : a \prec b,$ and $g : b \prec c$ we have an
  instance $g \circ f : a \prec c.$
\end{itemize}

Let us first justify the terminology, then the notation. So first consider a (set-theoretic) relation
$(\sim)$ on $S.$ Then for $\V := \Bool$ and
\[(a \prec b) := \begin{cases}
  \, 1 & \text{if } a \sim b,\\
  \, 0 & \text{otherwise}
\end{cases}\]
for $a,b \in S,$ $(\prec)$ is reflexive/symmetric/transitive whenever $(\sim)$ is.

The notation we use for the three specific instances can be understood category-theoretically:
Let $S$ be the set of objects in a category, and let $(a \to b)$ denote the set of morphisms
from $a \in S$ to $b \in S.$ Then the morphism arrow $(\to)$ is in fact a $\Set$-valued
meta-relation on $S,$ and all notations coincide:
\begin{itemize}
  \item A morphism $f : a \to b$ is indeed an instance of the type $(a \to b) \in \Set.$
  \item For each $a \in S,$ we have $\id_a : a \to a.$
  \item For an isomorphism $f : a \to b,$ we have $f^{-1} : b \to a.$
  \item For morphisms $f : a \to b$ and $g : b \to c,$ we have $g \circ f : a \to c.$
\end{itemize}

So morphisms in a category form a reflexive and transitive $\Set$-valued meta-relation.
Moreover, isomorphisms form a reflexive, symmetric, and transitive $\Set$-valued meta-relation.

Note, however, that in general we do not assume that $(\circ)$ is associative, that $\id_a$ is
an identity with respect to $(\circ),$ or that $f^{-1}$ is an inverse of $f.$ We will add such
assumptions later when needed, but in a more general form that avoids equality. For now, we
arbitrarily define the symbol ``$\circ$'' to be right-associative.

(See also \cite{HoTT-eq}.)

\subsection{Instance equivalences}

\begin{definition}
  From now on, we will assume every universe $\U = (I, (S_A)_{A \in I})$ to be equipped with an
  \defn{instance equivalence}, which we define to be
  \begin{itemize}
    \item a universe $\V,$ along with
    \item for each type $A \in \U,$ a reflexive, symmetric, and transitive $\V$-valued
    meta-relation $(\equiv)_A$ on $S_A.$ (In its infix form, we just write ``$\equiv$''.)
  \end{itemize}
\end{definition}

We say that ``$\U$ has instances equivalences in $\V$.''

The idea behind attaching an instance equivalence to a universe is that different universes
have different `natural' notions of equivalence of the instances of their
types.\footnote{In HoTT, we can assume that all of these instance equivalences are actually
equality.} Therefore, we will explicitly define instance equivalences for some, but not all,
of the examples given in section \ref{sec:universes}.

\begin{itemize}
  \item For every collection $\C$ of sets, the universe $\Set_\C$ has instance equivalences
  in $\Bool,$ by converting the set-theoretic equality relation on each set in $\C$ to a
  meta-relation as specified in the previous section.\\
  Note that the equivalences of both $\Bool$ and $\Unit$ are then actually in $\Unit$ (as a
  subuniverse of $\Bool$), as each type in $\Bool$ and $\Unit$ has at most one instance.
  \item Universes of algebraic structures (groups, rings, etc.)\ have the same instance
  equivalence as $\Set.$ (We may say that they inherit instance equivalences from $\Set$
  because their instances do not have any internal structure, and this idea also generalizes
  to structures built on universes other than $\Set.$)
  \item In the universe of categories, we define $a \equiv b$ to be the set of
  isomorphisms from $a$ to $b$, as described in the previous section. That is, the
  universe of categories has instance equivalences in $\Set.$\\(Similarly, higher categories
  generally have instance equivalences in some universe of categories or higher categories.
  We will not investigate this in detail, but it is likely that some concepts in this
  document correspond closely to higher category theory.)
  \item In section \ref{sec:equivalences}, we will give a definition of equivalence of types in
  a universe, and this will also be our definition of instance equivalences of universes of
  universes.
  \item Subuniverses inherit instance equivalences from their superuniverse.
  \item If $\U$ has instance equivalences in $\V,$ and $\U'$ has instance equivalences in
  $\V',$ then the product and sum universes $\U \times \U'$ and $\U \uplus \U'$ have instance
  equivalences in $\V \times \V'$ and $\V \uplus \V',$ respectively.
\end{itemize}

We will make sure that in the universes we deal with, for every sequence of universes
$\U_1, \U_2, \ldots,$ where each $\U_k$ has instance equivalences in $\U_{k+1},$ there is
a $k$ such that $\U_k = \U_{k+1} = \cdots = \Unit.$ However, at this point we do not consider
the interactions between the steps in such a sequence.\footnote{Readers familiar with HoTT may
have noticed that instance equivalences should have the structure of a higher groupoid that is
reflected in this sequence. However, we want to specify our assumptions in a more fine-grained
manner.}

\section{Functors}
\label{sec:functors}

The next piece of structure that we attach to universes -- and the first that lets us derive
some concrete results -- is a generalization of functions to what we call \defn{functors}.
Although functors between categories are indeed one special case of this definition, the
conditions are much weaker.

\begin{definition}
  For universes $\U = (I, (S_A)_{A \in I})$ and $\V = (J, (T_B)_{B \in J}),$ a
  \defn{functor type} from $A \in \U$ to $B \in \V$ is a type $(A \to B)$ in a universe $\W$
  with the following two properties:
  \begin{itemize}
    \item For every instance $F : A \to B$ (which we call a \defn{functor}) we have a
    function $\map_{ABF} : S_A \to T_B.$ Given $a : A,$ we will abbreviate
    ``$\map_{ABF}(a)$'' to ``$F(a).$''
    \item Moreover, $\map_{ABF}$ must respect instance equivalences: For each
    $a,b : A,$ we have a function $\congrArg_{ABFab}$ that maps instances of the type
    $a \equiv b$ to instances of $F(a) \equiv F(b).$\footnote{Although ideally we want
    $\congrArg$ to be a functor as well, recursively, at this point we do not make such an
    assumption.}
    For an equivalence $e : a \equiv b,$ we write ``$F(e)$'' for $\congrArg_{ABFab}(e)$ as
    well, matching the corresponding overloaded notation in category theory.
  \end{itemize}
  
  We say that we \defn{have functors from $\U$ to $\V$ in $\W$} if for every $A \in \U$ and
  $B \in \V$ we have a functor type $(A \to B) \in \W.$
  We say that a universe $\U$ has \defn{embedded functors} if we have functors from $\U$ to
  $\U$ in $\U.$

  As is common practice, we define the symbol ``$\to$'' to be right-associative, i.e.\ the
  notation ``$A \to B \to C$'' stands for ``$A \to (B \to C).$'' We call such a functor $F$ a
  \defn{bifunctor}, and we write ``$F(a,b)$'' for ``$F(a)(b).$'' (In section \ref{sec:products},
  we will identify $A \to B \to C$ with $A \times B \to C,$ where $A \times B$ is a product
  type.)
\end{definition}

The definition of functors is so generic that we can, in principle, define functors between
many different types in many different universes. However, universes with embedded functors
are much more rarer. Let us analyze a few examples.
\begin{itemize}
  \item The functors of $\Set$ are just functions, which respect equality and are obviously
  in $\Set$.
  \item The universe of categories has embedded functors: For categories $\C$ and $\D$, the
  (categorical) functors from $\C$ to $\D$ form a category $\D^\C,$ and we define the type
  $(\C \to \D)$ to be that category.\\
  We need to verify that functors respect instance equivalences. Recall that for objects
  $a$ and $b$ of either $\C$ or $\D,$ the type $a \equiv b$ is the set of isomorphisms from
  $a$ to $b.$ Indeed, functors map isomorphisms to isomorphisms.
  \item The same is true for the universe of groupoids, as a subuniverse of the universe of
  categories, because the category of functors between two groupoids is a groupoid. (For
  suitable definitions of instance equivalences, it should also generalize to higher
  categories and groupoids.)
  \item The morphisms of some, but not all, algebraic structures are also embedded functors
  in our sense: In some cases morphisms of a class of structures are themselves instances
  of that class of structures, when operations on morphisms are defined `pointwise'.

  For example, if $f,g : S \to T$ are two morphisms of commutative semigroups, then we can
  define $f \star g$ to be the function that sends each $a \in S$ to $f(a) \ast g(a).$ This
  is easily verified to be a morphism as well, based on associativity and commutativity
  of $(\ast).$ Moreover $(\star)$ inherits associativity and commutativity from $(\ast),$
  turning the set of morphisms from $S$ to $T$ into a semigroup.

  We may investigate the necessary and sufficient conditions more generally later, but for
  now, the following non-exhaustive list of structures with embedded functors will have to
  do:
  \begin{itemize}
    \item commutative semigroups, monoids, and groups
    \item modules over a ring
    \item vector spaces over a field
  \end{itemize}
  % TODO: Do commutative rings without 1 work?
  \item Continuous functions between topological spaces can be regarded as functors.
  \item The universe $\Unit$ only has a single type $1$, so we must set $(1 \to 1) := 1.$
  The type $1$ has exactly one instance $\emptyset$, so $\map$ is completely
  defined by $\map_{11\emptyset}(\emptyset) := \emptyset,$ and $\congrArg$ is completely
  defined by $\congrArg_{11\emptyset\emptyset\emptyset} := \emptyset.$
  \item For $\Bool,$ we set
  \begin{align*}
    (0 \to 0) &:= 1,\\
    (0 \to 1) &:= 1,\\
    (1 \to 0) &:= 0,\\
    (1 \to 1) &:= 1,
  \end{align*}
  matching logical implication.\footnote{This can be regarded as a degenerate case of
  the Curry-Howard correspondence.} Since $0$ has no instances and $1$ only has
  $\emptyset,$ we need to define three functions $\map_{00\emptyset},$
  $\map_{01\emptyset},$ and $\map_{11\emptyset}.$ The first two have empty domains,
  and the third is again completely defined by
  $\map_{11\emptyset}(\emptyset) := \emptyset.$
\end{itemize}

\begin{definition}
  For universes $\U = (I, (S_A)_{A \in I})$ and $\V = (J, (T_B)_{B \in J})$ with functors in
  $\W,$ types $A \in \U$ and $B \in \V,$ and a function $f : S_A \to T_B,$ we define the
  notation
  \begin{align*}
    F : A &\to     B\\
        a &\mapsto f(a)
  \end{align*}
  to mean that $F$ is a functor from $A$ to $B,$ and that for each $a : A$ we have an instance
  equivalence $\fndef{F}(a) : F(a) \equiv f(a).$ We call $\fndef{F}$ the
  \defn{definition of $F$}.

  We extend this notation to bifunctors: Given appropriate universes and types and an
  appropriate function $f,$ we define
  \begin{align*}
    F : A \to B &\to     C\\
        (a,b)   &\mapsto f(a,b)
  \end{align*}
  to mean that $F$ is a functor from $A$ to $(B \to C),$ and that for each $a : A$ and $b : B$ we
  have an instance equivalence $\fndef{F}(a,b) : F(a,b) \equiv f(a,b).$

  Likewise for \defn{trifunctors} $F : A \to B \to C \to D,$ and so on.
\end{definition}

Note that not every function $f$ gives rise to a functor, even though the notation might suggest
it (intentionally, as we will see). We will now assert the existence of certain functors
axiomatically.

\subsection{Functor operations}
\label{sec:funop}

\begin{definition}
  We say that a universe $\U$ with embedded functors has \defn{linear functor
  operations}\footnote{The terms ``linear'' and ``affine'' refer to linear and affine logic.}
  if we have the following three functors and six instance equivalences for all types
  $A,B,C,D \in \U.$
  \vspace{1ex}
  \begin{gather*}
    \begin{aligned}
      \idFun_A : A &\to     A\\
                 a &\mapsto a
    \end{aligned}\\[1em]
    \begin{aligned}
      \appFun_{AB} : A \to (A \to B) &\to     B\\
                     (a,F)           &\mapsto F(a)
    \end{aligned}\\[1em]
    \begin{aligned}
      \compFun_{ABC} : (A \to B) \to (B \to C) \to A &\to     C\\
                       (F,G,a)                       &\mapsto G(F(a))
    \end{aligned}
  \end{gather*}

  Before stating the required equivalences, we derive two additional functors from $\appFun$
  and $\compFun.$
  To improve readability, we want to use the symbol ``$\circ$'' when $\compFun$ is applied to
  two arguments, and indeed the functor arrow $(\to)$ is a $\U$-valued meta-relation on the
  set of types, which, given linear functor operations, is
  \begin{itemize}
    \item reflexive with $\id_A := \idFun_A$ and
    \item transitive with $G \circ F := \compFun_{ABC}(F,G)$ for $F : A \to B$ and
    $G : B \to C.$
  \end{itemize}

  Now we define
  \begin{align*}
    \swapFun_{ABC} &:= \compFun_{B((B{\to}C){\to}C)(A{\to}C)}(\appFun_{BC}) \circ \compFun_{A(B{\to}C)C}\\
                   &: \begin{aligned}[t]
                        (A \to B \to C) \to B \to A &\to     C\\
                        (F,b,a)                     &\mapsto F(a,b)
                      \end{aligned}
  \end{align*}
  and
  \begin{align*}
    \revCompFun_{ABC} &:= \swapFun(\compFun_{ABC})\\
                      &: \begin{aligned}[t]
                           (B \to C) \to (A \to B) \to A &\to     C\\
                           (G,F,a)                       &\mapsto G(F(a))
                         \end{aligned}
  \end{align*}
  (omitting the subscript of $\swapFun$ when applying it to an argument)
  and assert the existence of the following instance equivalences.
  %
  \begin{align*}
    \nm{rightId}_{AB} &: \compFun_{AAB}(\idFun_A) \equiv \idFun_{A{\to}B}\\
    \nm{leftId}_{AB}  &: \revCompFun_{ABB}(\idFun_B) \equiv \idFun_{A{\to}B}\\
    \nm{swapT}_{AB}   &: \swapFun(\appFun_{AB}) \equiv \idFun_{A{\to}B}\\
    \nm{swapB'}_{ABC} &: \swapFun_{(B{\to}C)AC} \circ \compFun_{ABC} \equiv \revCompFun_{AB((B{\to}C){\to}C)}(\appFun_{BC})\\
    \nm{swapB}_{ABC}  &: \swapFun_{(A{\to}B)AC} \circ \revCompFun_{ABC} \equiv \revCompFun_{A((A{\to}B){\to}B)((A{\to}B){\to}C)}(\appFun_{AB}) \circ \revCompFun_{(A{\to}B)BC}\\
    \nm{assoc}_{ABCD} &: \compFun_{(B{\to}C)((C{\to}D){\to}(B{\to}D))((C{\to}D){\to}(A{\to}D))}(\compFun_{BCD}) \circ \revCompFun_{(C{\to}D)(B{\to}D)(A{\to}D)} \circ \compFun_{ABD} \equiv\\
                      &\:\ \ \revCompFun_{(B{\to}C)}(\compFun_{ACD}) \circ \compFun_{ABC}
  \end{align*}
\end{definition}

\begin{remark}
  When defining $\swapFun$ and $\revCompFun,$ we implicitly assumed that we can derive
  instance equivalences
  \[\fndef{\swapFun}(F,b,a) : \swapFun(F,b,a) \equiv F(a,b) \quad \text{for } F : A \to B \to C,\ b : B,\ a : A\]
  and
  \[\fndef{\revCompFun}(G,F,a) : \revCompFun(G,F,a) \equiv G(F(a)) \quad \text{for } G : B \to C,\ F : A \to B,\ a : A.\]
  In order to obtain these, we first establish one further aspect in which functors behave like
  functions.
\end{remark}

\begin{proposition}
  \label{prp:congrfun}
  Given linear functor operations, two functors $F,G : A \to B,$ an instance equivalence
  $e : F \equiv G,$ and an instance $a : A,$ we can obtain an instance equivalence
  $e(a) : F(a) \equiv G(a).$
\end{proposition}
\vspace{-1ex}
\begin{proof}
  The functor $\appFun_{AB}(a)$ must respect instance equivalences, so we have an equivalence
  \[\appFun_{AB}(a)(e) : \appFun_{AB}(a,F) \equiv \appFun_{AB}(a,G).\]
  Applying the definition of $\appFun_{AB}$ yields an equivalence of the desired type:
  \begin{align*}
    e(a) &:= \fndef{\appFun_{AB}}(a,G) \circ \appFun_{AB}(a)(e) \circ (\fndef{\appFun_{AB}}(a,F))^{-1}\\
         &: F(a) \equiv G(a).
    \qedhere
  \end{align*}
\end{proof}

Together with the other properties of instance equivalences, this proposition establishes
that when constructing an instance equivalence involving functors, we may freely rewrite
along other equivalences, i.e.\ substitute arbitrary subterms. Still, the result will be an
explicit construction, which may be important if e.g.\ equivalences are
isomorphisms.\footnote{In terms of type theory, we are simply doing proof-relevant mathematics.}

We will usually avoid spelling out the details of such constructions,\footnote{Explicit
constructions of all equivalences can be found in the Lean formalization.} but for
demonstration purposes we will explicitly construct $\fndef{\swapFun}$ now.

Recall that ``$\circ$'' is just a shorthand for $\compFun$ in reverse order. So for a given
$F : A \to B \to C,$ the term $\swapFun(F)$ is exactly the left side of an equivalence given
by $\fndef{\compFun}$:
\begin{align*}
  e &:= \fndef{\compFun}(\compFun_{A(B{\to}C)C},\compFun_{B((B{\to}C){\to}C)(A{\to}C)}(\appFun_{BC}),F)\\
    &: \swapFun(F) \equiv \compFun_{A(B{\to}C)C}(F) \circ \appFun_{BC}.
\end{align*}
Applying proposition \ref{prp:congrfun} to $e$ and an instance $b : B,$ we obtain
\[e(b) : \swapFun(F,b) \equiv (\compFun_{A(B{\to}C)C}(F) \circ \appFun_{BC})(b).\]
The right side is again an application of $\compFun,$ so we have
\begin{align*}
  f &:= \fndef{\compFun}(\appFun_{BC},\compFun_{A(B{\to}C)C}(F),b)\\
    &: (\compFun_{A(B{\to}C)C}(F) \circ \appFun_{BC})(b) \equiv \appFun_{BC}(b) \circ F,
\end{align*}
and, applying transitivity of $(\equiv),$
\begin{align*}
  g &:= f \circ e(b)\\
    &: \swapFun(F,b) \equiv \appFun_{BC}(b) \circ F.
\end{align*}
Apply proposition \ref{prp:congrfun} to $g$ and an instance $a : A$ to obtain
\[g(a) : \swapFun(F,b,a) \equiv (\appFun_{BC}(b) \circ F)(a).\]
This time we have two relevant definitions
\begin{align*}
  h &:= \fndef{\compFun}(F,\appFun_{BC}(b))\\
    &: (\appFun_{BC}(b) \circ F)(a) \equiv \appFun_{BC}(b,F(a)),\\[1ex]
  i &:= \fndef{\appFun}(b,F(a))\\
    &: \appFun_{BC}(b,F(a)) \equiv F(a,b).
\end{align*}
Finally, we can apply transitivity twice to arrive at
\begin{align*}
  \fndef{\swapFun}(F,b,a) &:= i \circ h \circ g(a)\\
                          &: \swapFun(F,b,a) \equiv F(a,b).
\end{align*}

\begin{remarks}
  For a bifunctor $F : A \to B \to C,$ the bifunctor $\swapFun(F) : B \to A \to C$ behaves
  (up to instance equivalences) like $F$ with swapped arguments. So, informally speaking,
  a bifunctor is a bifunctor regardless of the order of its arguments.

  This should also help clarify the role of $\compFun$ and $\revCompFun.$ The existence
  of $\compFun$ ensures that we can not only compose two functors $F : A \to B$ and
  $G : B \to C$ to $G \circ F : A \to C,$ but also that composition itself is bifunctorial
  in $F$ and $G.$ $\revCompFun,$ then, is the corresponding bifunctor with reversed
  arguments, and in fact we could assert $\revCompFun$ as an axiom and derive $\compFun$
  from it instead.\footnote{Due to a minor technical detail, $\compFun$ leads to a slightly
  more convenient definition of $\swapFun.$}

  Similarly $\appFun$ says that for each $a : A,$ the application of $a$ to an
  $F : A \to B$ is functorial, and also that this application functor is functorial in
  $a.$

  Moreover, all of these functors can in fact be regarded as combinators\cite{Combinators}
  in a simply-typed lambda calculus\cite{Simply-typed lambda calculus}, so we use
  established names for these combinators as much as possible.
\end{remarks}

\vspace{1ex}
\begin{definition}
  We say that a universe $\U$ with embedded functors has \defn{affine functor
  operations} if it has linear functor operations and additionally the following functor
  and equivalences for all types $A,B,C \in \U.$
  \begin{align*}
    \constFun_{AB} : B \to A &\to     B\\
                     (b,a)   &\mapsto b
  \end{align*}
  \begin{align*}
    \nm{rightConst}_{ABC} &: \compFun_{B(A{\to}B)(A{\to}C)}(\constFun_{AB}) \circ \revCompFun_{ABC} \equiv \revCompFun_{BC(A{\to}C)}(\constFun_{AC})\\
    \nm{leftConst}_{ABC}  &: \compFun_{C(B{\to}C)(A{\to}C)}(\constFun_{BC}) \circ \compFun_{ABC} \equiv \constFun_{(A{\to}B)(C{\to}A{\to}C)}(\constFun_{AC})
  \end{align*}
\end{definition}

\begin{definition}
  We say that a universe $\U$ with embedded functors has \defn{full functor
  operations} if it has affine functor operations and additionally the following functor
  and equivalences for all types $A,B,C \in \U.$

  First we assert the existence of
  \begin{align*}
    \dupFun_{AB} : (A \to A \to B) \to A &\to     B\\
                   (F,a)                 &\mapsto F(a,a)
  \end{align*}
  and derive
  \begin{align*}
    \substFun_{ABC} &:= \revCompFun_{(A{\to}B{\to}C)(A{\to}A{\to}C)(A{\to}C)}(\dupFun_{AC}) \circ
                        \revCompFun_{A(B{\to}C)(A{\to}C)} \circ
                        \compFun_{ABC}\\
                    &: \begin{aligned}[t]
                         (A \to B) \to (A \to B \to C) \to A &\to     C\\
                         (F,G,a)                             &\mapsto G(a,F(a))
                       \end{aligned}
  \end{align*}
  and
  \begin{align*}
    \revSubstFun_{ABC} &:= \swapFun(\substFun_{ABC})\\
                       &: \begin{aligned}[t]
                            (A \to B \to C) \to (A \to B) \to A &\to     C\\
                            (G,F,a)                             &\mapsto G(a,F(a)).
                          \end{aligned}
  \end{align*}

  Then we assert the existence of the following equivalences.
  \begin{align*}
    \nm{dupC}_{AB}      &: \dupFun_{AB} \circ \swapFun_{AAB} \equiv \dupFun_{AB}\\
    \nm{dupK}_{AB}      &: \dupFun_{AB} \circ \constFun_{A(A{\to}B)} \equiv \idFun_{A{\to}B}\\
    \nm{dupW}_{AB}      &: \dupFun_{AB} \circ \dupFun_{A(A{\to}B)} \equiv \dupFun_{AB} \circ \revCompFun_{A(A{\to}A{\to}B)(A{\to}B)}(\dupFun_{AB})\\
    \nm{rightDup}_{ABC} &: \begin{aligned}[t]\compFun_{ABC} \circ \dupFun_{AB} &\equiv \revCompFun_{(B{\to}C)(A{\to}A{\to}C)(A{\to}C)}(\dupFun_{AC})\,\circ\\
                                                                               &\ \ \ \ \compFun_{(B{\to}C)((A{\to}B){\to}(A{\to}C))(A{\to}A{\to}C)}(\revCompFun_{ABC}) \circ \compFun_{A(A{\to}B)(A{\to}C)}\end{aligned}\\
    \nm{leftDup}_{ABC}  &: \compFun_{(A{\to}B{\to}B{\to}C)(A{\to}B{\to}C)(A{\to}C)}(\revCompFun_{A(B{\to}B{\to}C)(B{\to}C)}(\dupFun_{BC})) \circ \substFun_{ABC} \equiv\\
                        &\:\ \ \substFun(\substFun_{ABC},\compFun_{(A{\to}B{\to}B{\to}C)(A{\to}B{\to}C)(A{\to}C)} \circ \substFun_{AB(B{\to}C)})
  \end{align*}
\end{definition}

\begin{lemma}
  \label{lem:extensional}
  The axioms $\nm{rightId},\nm{leftId},\ldots$ hold trivially (whenever the functors
  referenced in those axioms are defined) in any universe that satisfies the following
  extensionality condition:

  If $A$ and $B$ are types, $F,G : A \to B$ are functors, and for every $a : A$ we have
  an equivalence $e(a) : F(a) \equiv G(a),$ then there is also an equivalence
  $e : F \equiv G.$
\end{lemma}

(Note that proposition \ref{prp:congrfun} says that the converse is always true in
a universe with linear functor operations.)

\begin{proof}
  Under the extensionality condition, the equivalence
  \[\nm{rightId}_{AB} : \compFun_{AAB}(\idFun_A) \equiv \idFun_{A{\to}B}\]
  exists if for every $F : A \to B$ we have an equivalence
  \[\nm{rightId}_{AB}(F) : F \circ \idFun_A \equiv \idFun_{A{\to}B}(F).\]
  By straightforward operations on equivalences and another application of the
  extensionality condition, this equivalence exists if for every $a : A$ we have an
  equivalence
  \[\nm{rightId}_{AB}(F,a) : F(a) \equiv F(a),\]
  which is given by $\id_{F(a)}.$

  The other axioms are analogous.
\end{proof}

\begin{theorem}
  \label{thm:funop}
  The universes with embedded functors that are listed in section \ref{sec:functors} have the
  following functor operations.
  \begin{itemize}
    \item $\Unit,$ $\Bool,$ $\Set,$ and the universes of categories and groupoids have
    full functor operations.
    \item The listed universes of algebraic structures have linear functor operations.
  \end{itemize}
\end{theorem}

\begin{proof}
  We will give explicit proofs for some important cases; the remaining axioms and
  universes are analogous.
  %
  \begin{itemize}
    \item In $\Unit,$ we take each functor to be the single instance $\emptyset : 1.$
    All axioms are trivially satisfied.
    %
    \item For $\Bool,$ we can show that all functors exist by doing a case-by-case
    analysis on the types $A,B,\ldots$ being either $0$ or $1.$ This can be simplified by
    treating ``$\to$'' as implication and observing that all implications hold, or even
    further by applying the Curry-Howard correspondence\cite{Curry-Howard}.\\
    Since $\Bool$ has instance equivalences in $\Unit,$ those are trivially satisfied.
    %
    \item The functors of $\Set$ are just functions. Since functions are
    extensional,\footnote{The theory can also be formalized in a logic without function
    extensionality by \emph{defining} functors between sets to be extensional functions.
    Alternatively/additionally, it is possible to define a universe of setoids, which
    is similar to $\Set$ except that equality is replaced with an equivalence relation.}
    the axioms $\nm{rightId},\nm{leftId},\ldots$ hold by lemma \ref{lem:extensional}.
    %
    \item To show how to construct the required functors in universes of
    structures, we will take $\constFun_{\C\D}$ for categories $\C$ and $\D$ as a
    simple but not completely trivial example.

    The definition of $\constFun_{\C\D}$ says that for objects $c$ of $\C$ and $d$ of
    $\D,$ $\constFun_{\C\D}(d,c)$ must be isomorphic to $d.$ In many cases, we do not
    actually need this flexibility; we can construct a functor that maps exactly to
    $d.$ That is, we can take this expression at face value and show that it is
    functorial in all arguments starting from the last.\footnote{In the Lean
    formalization, the axioms are already divided into such individual steps, which is
    often more useful.}

    So the first step is to show that for a fixed object $d$ of $\D$ we have a functor
    \begin{align*}
      K_d : \C &\to     \D\\
            c  &\mapsto d,
    \end{align*}
    i.e. for objects $c,c'$ of $\C$ and a morphism $f : c \to c'$ we need to
    provide a morphism $K_d(f) : K_d(c) \to K_d(c').$ But $K_d(c)$ and $K_d(c')$
    are both $d,$ so we can define $K_d(f)$ to be the identity morphism on $d.$ (Then
    $K_d$ is just the constant functor, of course.)

    The second step is to show that the expression $K_d$ is functorial in $d,$ and
    this will give the desired functor
    \begin{align*}
      \constFun_{\C\D} : \D &\to     \D^\C\\
                         d  &\mapsto K_d.
    \end{align*}
    For objects $d,d'$ of $\D$ and a morphism $f : d \to d',$ we need to provide a
    natural transformation $\constFun_{\C\D}(f) : K_d \Rightarrow K_{d'}.$
    Thus, for each object $c$ of $\C$ we need to give a morphism
    $g_c : K_d(c) \to K_{d'}(c).$ Since $K_d(c) = d$ and $K_{d'}(c) = d',$ we can
    take $g_c := f.$ ($\constFun_{\C\D}$ is known as the diagonal functor.)
    %
    \item The least straightforward case is the construction of $\dupFun_{\C\D}$ for
    categories $\C$ and $\D.$ Following the same strategy as before, first we fix a
    functor $F : \C \to \D^\C$ and need to construct a functor
    \begin{align*}
      W_F : \C &\to     \D\\
            c  &\mapsto F(c)(c).
    \end{align*}
    So for objects $c,c'$ of $\C$ and a morphism $f : c \to c'$ we need to provide
    a morphism $W_F(f) : F(c)(c) \to F(c')(c').$ Since $F(f)$ is a natural
    transformation, the two choices for this morphism are equal:
    \[W_F(f) := (F(f))_{c'} \circ F(c)(f) = F(c')(f) \circ (F(f))_c.\]
    $W_F$ is indeed a functor: We have
    \[W_F(\id_c) = (\id_{F(c)})_{c} \circ \id_{F(c)(c)} = \id_{F(c)(c)}\]
    and for morphisms $f : c \to c'$ and $g : c' \to c''$ in $\C$
    \begin{align*}
      W_F(g \circ f) &= (F(g \circ f))_{c''} \circ F(c)(g \circ f)\\
                     &= (F(g))_{c''} \circ (F(f))_{c''} \circ F(c)(g) \circ F(c)(f)\\
                     &= (F(g))_{c''} \circ F(c')(g) \circ (F(f))_{c'} \circ F(c)(f) \quad \text{by naturality of $F(f)$}\\
                     &= W_F(g) \circ W_F(f).
    \end{align*}
    Now we need to show that $W_F$ is functorial in $F$ to obtain
    \begin{align*}
      \dupFun_{\C\D} : (\D^\C)^\C &\to     \D^\C\\
                       F          &\mapsto W_F.
    \end{align*}
    Given two functors $F,F' : \C \to \D^C$ and a natural transformation
    $\eta : F \Rightarrow F',$ we need to provide a natural transformation
    $\dupFun_{\C\D}(\eta) : W_F \Rightarrow W_{F'}.$ We set
    $(\dupFun_{\C\D}(\eta))_c := (\eta_c)_c$ for each object $c$ of $\C,$
    and need to verify that this is natural in $c.$ Indeed, for every
    morphism $f : c \to c'$ we have
    \begin{align*}
      (\eta_{c'})_{c'} \circ W_F(f) &= (\eta_{c'} \circ F(f))_{c'} \circ F(c)(f)\\
                                    &= (F'(f) \circ \eta_c)_{c'} \circ F(c)(f) \quad \text{by naturality of $\eta$}\\
                                    &= (F'(f))_{c'} \circ (\eta_c)_{c'} \circ F(c)(f)\\
                                    &= (F'(f))_{c'} \circ F'(c)(f) \circ (\eta_c)_c \quad \text{by naturality of $\eta_c$}\\
                                    &= W_{F'}(f) \circ (\eta_c)_c.
    \end{align*}
    It is easily verified that $\dupFun_{\C\D}$ respects identity and
    composition of natural transformations.
    \item Although lemma \ref{lem:extensional} does not apply to the universe of
    categories, the proof strategy for $\nm{rightId},\ldots$ is the same
    as in the proof of that lemma. The difference is that at each step where we
    would apply the extensionality condition, instead we need to verify that the
    equivalences $e(a)$ are natural in $a.$ \qedhere
  \end{itemize}
\end{proof}

\begin{conjecture}
  Theorem \ref{thm:funop} generalizes at least to $n$-groupoids and possibly
  also to $n$-categories.
\end{conjecture}

\begin{conjecture}
  Topological spaces satisfying some mild conditions also have full functor operations.
\end{conjecture}

\subsection{Functoriality algorithm}

After having shown that several important universes do in fact have linear or even full
functor operations, we will now describe an algorithm to prove functoriality
automatically, i.e.\ to obtain a functor that matches a given definition.

This algorithm is actually just a slight adaptation of the well-known algorithm to
transform lambda abstractions into terms built from combinators\cite{Combinators}. In a
simply-typed lambda calculus, this algorithm always terminates. So as long as one is
not interested in the specific behavior of the functor (beyond how it maps instances),
there is no need to execute the algorithm explicitly -- verifying the preconditions
given in proposition \ref{prp:functoriality} is sufficient.

First, we reduce definitions of bifunctors to definitions of functors; and analogously
trifunctors to bifunctors, and so on. This principle closely resembles the proof
strategy in theorem \ref{thm:funop}. Given a definition of a bifunctor
\begin{align*}
  F : A \to B &\to     C\\
      (a,b)   &\mapsto f(a,b),
\end{align*}
first recursively apply the functoriality algorithm to
\begin{align*}
  F_a : B &\to     C\\
        b &\mapsto f(a,b)
\end{align*}
for fixed $a : A,$ to obtain the functor $F_a : B \to C$ and an equivalence
$\fndef{F_a}(b) : F_a(b) \equiv f(a,b)$ for each $b : B.$ Then apply the
functoriality algorithm to
\begin{align*}
  F' : A &\to     (B \to C)\\
       a &\mapsto F_a
\end{align*}
to obtain $F' : A \to B \to C$ and an equivalence $\fndef{F'}(a) : F'(a) \equiv F_a$
for each $a : A.$
Finally set $F := F',$ apply proposition \ref{prp:congrfun} to obtain an equivalence
$\fndef{F'}(a)(b) : F(a,b) \equiv F_a(b)$, and set
$\fndef{F}(a,b) := \fndef{F_a}(b) \circ \fndef{F'}(a)(b).$

Due to this reduction, we can limit ourselves to the simple case
\begin{align*}
  F : A &\to     B\\
      a &\mapsto f(a)
\end{align*}
and perform a (non-exhaustive and non-unique) case split on $f.$

\begin{longtable}{ll|l}
  Definition & & $F$\\
  \hline\hline
  \rule{0pt}{2em}
  $\begin{aligned}
    F : A &\to     B\\
        a &\mapsto b
  \end{aligned}$ &
  \parbox{3cm}{for $b : B$ (constant \\ with respect to $a$)} &
  $\constFun_{AB}(b)$\\
  \hline
  \rule{0pt}{2em}
  $\begin{aligned}
    F : A &\to     A\\
        a &\mapsto a
  \end{aligned}$ & &
  $\idFun_A$\\
  \hline
  \rule{0pt}{2em}
  $\begin{aligned}
    F : A &\to     B\\
        a &\mapsto G(a)
  \end{aligned}$ &
  for $G : A \to B$ &
  $G$\\
  \hline
  \rule{0pt}{2em}
  $\begin{aligned}
    F : A &\to     C\\
        a &\mapsto G(b_a)
  \end{aligned}$ &
  \parbox{3cm}{for $b_a : B$ \\ and $G : B \to C$} &
  $\compFun_{ABC}(H,G)$ with
  $\begin{aligned}
    H : A &\to     B\\
        a &\mapsto b_a
  \end{aligned}$\\
  \hline
  \rule{0pt}{2em}
  $\begin{aligned}
    F : (B \to C) &\to     C\\
        G         &\mapsto G(b)
  \end{aligned}$ &
  for $b : B$ &
  $\appFun_{BC}(b)$\\
  \hline
  \rule{0pt}{2em}
  $\begin{aligned}
    F : A &\to     C\\
        a &\mapsto G_a(b)
  \end{aligned}$ &
  \parbox{3cm}{for $b : B$ \\ and $G_a : B \to C$} &
  $\swapFun_{ABC}(G,b)$ with
  $\begin{aligned}
    G : A &\to     (B \to C)\\
        a &\mapsto G_a
  \end{aligned}$\\
  \hline
  \rule{0pt}{2em}
  $\begin{aligned}
    F : A &\to     B\\
        a &\mapsto G_a(a)
  \end{aligned}$ &
  for $G_a : A \to B$ &
  $\dupFun_{AB}(G)$ with
  $\begin{aligned}
    G : A &\to     (A \to B)\\
        a &\mapsto G_a
  \end{aligned}$\\
  \hline
  \rule{0pt}{3em}
  $\begin{aligned}
    F : A &\to     C\\
        a &\mapsto G_a(b_a)
  \end{aligned}$ &
  \parbox{3cm}{for $b_a : B$ \\ and $G_a : B \to C$} &
  \parbox{6cm}{
    $\substFun_{ABC}(H,G)$ with \\[1ex]
    $\begin{aligned}
      H : A &\to     B\\
          a &\mapsto b_a
    \end{aligned}$
    and
    $\begin{aligned}
      G : A &\to     (B \to C)\\
          a &\mapsto G_a
    \end{aligned}$
  }
\end{longtable}

In all cases except the first two, $f(a)$ is a functor application. In fact, the
last case is the most general possible functor application, and in a universe with
full functor operations, all other cases of functor applications may be regarded as
mere optimizations.\\
Note that a functor application with multiple arguments $F(a_1,\ldots,a_n)$ is
really an application of the functor $F(a_1,\ldots,a_{n-1})$ to the argument $a_n,$
and must be treated as such.

One piece of information missing from the table is that the algorithm must produce
not only the functor $F$ but also an equivalence
$\fndef{F}(a) : F(a) \equiv f(a)$ for each $a : A.$ This equivalence is obtained by
composing the definition of the combinator with the definitions of the recursively
generated functors (if any).

The algorithm may produce terms of the form $\swapFun(\compFun,\ldots)$ or
$\swapFun(\substFun,\ldots).$ By the definitions of $\revCompFun$ and $\revSubstFun,$
these can be replaced with $\revCompFun(\ldots)$ and $\revSubstFun(\ldots),$
respectively.

\begin{proposition}
  \label{prp:functoriality}
  Let $\U$ be a universe with embedded functors and (at least) linear functor
  operations, and $A_1,\ldots,A_n$ and $B$ be types of $\U.$ Consider the (attempted)
  functor definition
  \begin{align*}
    F : A_1 \to \ldots \to A_n &\to     B\\
        (a_1,\ldots,a_n)       &\mapsto t_{a_1,\ldots,a_n}
  \end{align*}
  with a term\footnote{Formally, the attempted functor definition is given by a
  function $f : S_{A_1} \times \ldots \times S_{A_n} \to S_B,$ where $S_A$ is the
  set of instances of $A.$ However, the proposition is much more readable in its
  current informal form.} $t_{a_1,\ldots,a_n}$ that is one of the following:
  \begin{itemize}
    \item a constant independent of all $a_k,$
    \item one of the variables $a_k,$ or
    \item a functor application $G(b)$ where both $G$ and $b$ are terms that
    recursively follow the same rules.
  \end{itemize}
  Then the functoriality algorithm produces a functor matching the definition of $F$
  under the following additional constraints:
  \begin{itemize}
    \item If $\U$ only has linear functor operations (but does not have affine
    functor operations), each variable $a_k$ must occur exactly once in $t_{a_1,\ldots,a_n}.$
    \item If $\U$ only has affine functor operations (but does not have full
    functor operations), each variable $a_k$ must occur at most once in $t_{a_1,\ldots,a_n}.$
  \end{itemize}
\end{proposition}

\begin{proof}
  The algorithm given above is total under these conditions.
\end{proof}

\begin{example}
  Let us consider the simple case where we want to compose a bifunctor
  $F : A \to B \to C$ with a functor $G : C \to D.$ This can be understood in
  two different ways: We can either construct this composition for fixed but
  arbitrary $F$ and $G,$ or we can define it as a functor taking $F$ and $G$ as
  arguments.
  
  For fixed $F$ and $G,$ the functor we want to construct is
  \begin{align*}
    H_{FG} : A \to B &\to     D\\
             (a,b)   &\mapsto G(F(a,b)).
  \end{align*}
  The term $G(F(a,b))$ only consists of functor applications, references to the
  constants $F$ and $G,$ and references to the variables $a$ and $b,$ so we
  know that the functoriality algorithm can produce a functor matching this
  definition. Since each variable occurs in this term exactly once, the
  definition is valid in every universe with linear functor operations. If we
  actually execute the algorithm, we find that it outputs
  \[H_{FG} := \revCompFun_{BCD}(G) \circ F\]
  and
  \begin{align*}
    \fndef{H_{FG}}(a,b) &:= \fndef{\revCompFun_{BCD}}(G,F(a),b) \circ \fndef{\compFun_{A(B{\to}C)(B{\to}D)}}(F,\revCompFun_{BCD}(G),a)(b)\\
                        &: H_{FG}(a,b) \equiv G(F(a,b)).
  \end{align*}

  We can now interpret this construction in specific universes, for example:
  \begin{itemize}
    \item In the universe of categories:\\
    If $\A,\B,\C,\D$ are categories and $F : \A \to \C^\B$ and $G : \C \to \D$
    are functors, then we have a functor $H : \A \to \D^\B$ such that
    $H(a)(b)$ is isomorphic to $G(F(a)(b))$ for objects $a$ of $\A$ and $b$ of
    $\B.$
    \item In the universe of vector spaces over a field:\\
    If $V,W,X,Y$ are vector spaces over $K,$ $f$ is a linear map from $V$
    to the space of linear maps from $W$ to $X,$ and $g$ is a linear map
    from $X$ to $Y,$ then we have a linear map $h$ from $V$ to the space of
    linear maps from $W$ to $Y$ such that $h(v)(w) = g(f(v)(w))$ for vectors
    $v$ of $V$ and $w$ of $W.$
  \end{itemize}

  If instead we want to construct a functor that takes $F$ and $G$ as
  arguments, we can either execute the functoriality algorithm directly for
  \begin{align*}
    H : (A \to B \to C) \to (C \to D) \to A \to B &\to     D\\
        (F,G,a,b)                                 &\mapsto G(F(a,b)),
  \end{align*}
  or we can use the previous result and construct
  \begin{align*}
    H : (A \to B \to C) \to (C \to D) &\to     (A \to B \to D)\\
        (F,G)                         &\mapsto H_{FG} = \revCompFun_{BCD}(G) \circ F.
  \end{align*}
  The algorithm (always) produces the same result in both cases, which is
  \[H := \compFun_{(C{\to}D)((B{\to}C){\to}(B{\to}D))(A{\to}B{\to}D)}(\revCompFun_{BCD}) \circ \compFun_{A(B{\to}C)(B{\to}D)}.\]
\end{example}

\begin{remark}
  The utility of the functoriality algorithm can be a bit subtle, especially when dealing
  with concrete universes such as the universe of categories. As a rule of thumb, it can
  be used to prove functoriality of terms where all of the objects that appear in the term
  are already functorial. Or, from a reverse point of view, it eliminates the need to
  compose and otherwise manipulate functors explicitly -- instead, functors may simply be
  written as expressions that specify how to map objects, leaving the rest implicit.
\end{remark}

\subsection{Functor universe}
\label{sec:functor-universe}

In the universes we covered so far, the relationship between a type and the set of its
instances was always very straightforward (although they are truly equal only in $\Set_\C$).
In the universe that we are going to define now, we will need to be more careful about the
distinction: Recall that a type is just a member of an index set, and its instances are
actually defined by a family of sets that are indexed by types.

\begin{definition}
  Let $\U$ and $\V = (I, (S_B)_{B \in I})$ and $\W = (J, (T_C)_{C \in J})$ be universes
  such that we have functors from $\U$ to $\V$ in $\W,$ and let $A \in \U.$
  We define the \defn{functor universe} $\V^A$ to be the pair
  \[\V^A := (I, (T_{A \to B})_{B \in I}).\]
  That is,
  \begin{itemize}
    \item the types (or type indices, to be explicit) of $\V^A$ are the same as the types
          of $\V,$ but
    \item the instances in $\V^A$ of a type $B \in \V$ are actually functors from $A$ to $B.$
  \end{itemize}
  We let $\V^A$ inherit instance equivalences from $\W.$
\end{definition}

If types $B,C,\ldots \in \V$ are also types of $\V^A$ and vice versa, we need to be more
explicit about the meaning of instance declarations $b : B$ and also of functor types
$B \to C.$ To avoid having to annotate all symbols with universes, we adopt the purely
notational convention that $B^A$ is technically defined to be equal to $B \in \V$ but
should be understood as a type in $\V^A.$

So ``$b : B$'' is always an abbreviation for ``$b :_\V B,$'' whereas ``$F : B^A$'' is an
abbreviation for ``$F :_{\V^A} B,$'' so that $F$ is a functor from $A$ to $B$ even
though $B^A = B.$

\subsubsection*{Full functor operations}

We will first concentrate on the case where we have a single universe $\U$ with embedded
functors and full functor operations, and a type $A \in \U.$

\begin{proposition}
  In this case, $\U^A$ also has embedded functors.
\end{proposition}
\vspace{-1ex}
\begin{proof}
  For $B,C \in \U$ we define $(B^A \to C^A) := (B \to C)^A \in \U^A.$ For a functor
  $G : B^A \to C^A,$ which is also an instance of the type $A \to B \to C,$ and an
  instance $F : B^A,$ which is also an instance of $A \to B,$ we define their functor
  application in $\U^A$ by $G(F) := \revSubstFun_{ABC}(G,F).$ This definition
  respects instance equivalences because it is an application of the functor
  $\revSubstFun_{ABC}(G).$
\end{proof}

\begin{proposition}
  $\U$ embeds into $\U^A$: For each type $B \in \U$ and instance $b : B,$ we have
  a corresponding instance $b^A := \constFun_{AB}(b) : B^A.$ This embedding
  respects instance equivalences and functor application, up to instance equivalence.
\end{proposition}
\vspace{-1ex}
\begin{proof}
  As a functor, $\constFun_{AB}$ respects instance equivalences.
  Moreover, from $\nm{rightConst}$ and $\nm{dupK}$ we can derive an equivalence
  \[\nm{embedMap}(F,b) : F^A(b^A) \equiv (F(b))^A\]
  for every $F : B \to C$ and $b : B$ with $B,C \in \U.$
  (Note that the functor application on the left side is defined as
  $F^A(b^A) = \revSubstFun_{ABC}(\constFun_{A(B{\to}C}(F),\constFun_{AB}(b)).$)
\end{proof}

\begin{proposition}
  $\U^A$ has full functor operations. 
\end{proposition}
\vspace{-1ex}
\begin{proof}
  TODO
\end{proof}

Thus, the functor universe $\U^A$ inherits all properties from $\U,$ but it
additionally comes with a specific instance $\idFun_A : A^A.$ Intuitively, this
instance can be understood as a free variable $a : A$ that acts like any other term
of type $A.$ So we can regard $\U^A$ as the universe in which the `body' $t_a$ of a
functor definition
\begin{align*}
  F : A &\to     B\\
      a &\mapsto t_a
\end{align*}
lives:
In the previous section, we regarded $t_a : B$ as a term in $\U$ that depends on
$a : A.$ But we can also lift $t_a$ to a term $T_G : B^A$ depending on a $G : A^A,$
which gives a functor definition for the embedded functor
\begin{align*}
  F^A : A^A &\to     B^A\\
        G   &\mapsto T_G.
\end{align*}
Note that if $t_a$ is a functor application (in $\U$), then $T_G$ is a functor
application in $\U^A,$ which is given by $\revSubstFun.$ Since (assuming full functor
operations), all functor applications can be handled by $\revSubstFun$ in the
functoriality algorithm, we can regard the lifting operation and the functoriality
algorithm as two different descriptions of the same principle.

If we set $G := \idFun_A,$ then $T_G$ is equivalent to $F.$\footnote{This is quite
similar to how evaluating a polynomial $f \in R[X]$ at $X$ yields $f$ itself.}
More succinctly, we have the following

\begin{proposition}
  For every $B \in \U$ and $F : A \to B$ we have an equivalence
  \[\nm{embedId}(F) : F^A(\idFun_A) \equiv F.\]
\end{proposition}
\vspace{-2ex}
\begin{proof}
  By $\nm{rightConst},$ $\nm{dupK},$ and $\nm{rightId}.$
\end{proof}

\subsubsection*{Linear and affine functor operations}

TODO

\subsection{Extensionality}
\label{sec:extensionality}

\begin{theorem}
  \label{thm:extensionality}
  Functors in a universe $\U$ with full functor operations are ``extensional in practice:''
  If, for types $A,B \in \U$ and functors $F,G : A \to B,$ we can derive an equivalence
  $e(a) : F(a) \equiv G(a)$ purely from generic operations on equivalences and the
  axioms given in section \ref{sec:funop}, for a fixed but arbitrary $a : A,$ then we
  also have an equivalence $e : F \equiv G.$
\end{theorem}

\begin{proof}
  If the term $e(a)$ is composed purely of the generic operations and axioms that we
  defined, then it can be lifted to an equivalence $e^A(H) : F^A(H) \equiv G^A(H)$ in the
  functor universe $\U^A,$ dependent on an arbitrary functor $H : A^A.$ Now set
  \begin{align*}
    e &:= \nm{embedId}(G) \circ e^A(\idFun_A) \circ (\nm{embedId}(F))^{-1}\\
      &: F \equiv G. \qedhere
  \end{align*}
\end{proof}

\begin{conjecture}
  \label{conj:extensionality}
  The above theorem extends to universes with linear or affine functor operations.
\end{conjecture}

Due to the way in which the extensionality axioms are split between the three types of
functor operations, this conjecture is almost guaranteed to be true. However, without
full functor operations we cannot simply embed everything into the functor universe.
A possible alternative might be to work with the sum universe $\U \uplus \U^A \uplus \Set_{\{0\}}.$

\begin{remarks}
  * Efficiency
  * More direct algorithm
  * Further structure
  * Significance regarding natural isomorphisms

  A list of axioms that ensure extensionality in $\nm{SKI}$ combinator calculus is given
  in \cite{Extensionality}.
\end{remarks}

\begin{corollary}
  Whenever the functoriality algorithm offers multiple alternatives, there are instance
  equivalences between the resulting functors (in a universe with full functor
  operations, or unconditionally assuming conjecture \ref{conj:extensionality}).
\end{corollary}

\begin{remark}
  The following alternatives exist. (Brackets indicate that an alternative is redundant
  because it can also be regarded as an alternative of one or more other alternatives.)

  \begin{longtable}{ll|l}
    Case & & Alternatives\\
    \hline\hline
    \rule{0pt}{3em}
    $\begin{aligned}
      F : A &\to     C\\
          a &\mapsto G(b)
    \end{aligned}$ &
    \parbox{3cm}{for $b : B$ \\ and $G : B \to C$} &
    $\begin{array}{@{}l@{}}
      \constFun_{AC}(G(b))\\
      \compFun_{ABC}(\constFun_{AB}(b),G)\\
      \swapFun_{ABC}(\constFun_{A(B{\to}C)}(G),b)\\
      \left[\substFun_{ABC}(\constFun_{AB}(b),\constFun_{A(B{\to}C)}(G))\right]
    \end{array}$\\
    \hline
    \rule{0pt}{3em}
    $\begin{aligned}
      F : A &\to     B\\
          a &\mapsto G(a)
    \end{aligned}$ &
    for $G : A \to B$ &
    $\begin{array}{@{}l@{}}
      G\\
      \compFun_{AAB}(\idFun_A,G)\\
      \dupFun_{AB}(\constFun_{A(A{\to}B)}(G))\\
      \left[\substFun_{AAB}(\idFun_A,\constFun_{A(A{\to}B)}(G))\right]
    \end{array}$\\
    \hline
    \rule{0pt}{2em}
    $\begin{aligned}
      F : A &\to     C\\
          a &\mapsto G(b_a)
    \end{aligned}$ &
    \parbox{3cm}{for $G : B \to C$ \\ and $b_a : B$} &
    $\begin{array}{@{}l@{}}
      \compFun_{ABC}(H,G)\\
      \substFun_{ABC}(H,\constFun_{B(B{\to}C)}(G))
    \end{array}$
    with
    $\begin{aligned}
      H : A &\to     B\\
          a &\mapsto b_a
    \end{aligned}$\\
    \hline
    \rule{0pt}{2.5em}
    $\begin{aligned}
      F : (B \to C) &\to     C\\
          G         &\mapsto G(b)
    \end{aligned}$ &
    for $b : B$ &
    $\begin{array}{@{}l@{}}
      \appFun_{BC}(b)\\
      \swapFun_{(B{\to}C)BC}(\idFun_{B{\to}C},b)\\
      \left[\substFun_{(B{\to}C)BC}(\constFun_{(B{\to}C)B}(b),\idFun_{B{\to}C})\right]
    \end{array}$\\
    \hline
    \rule{0pt}{2em}
    $\begin{aligned}
      F : A &\to     C\\
          a &\mapsto G_a(b)
    \end{aligned}$ &
    \parbox{3cm}{for $b : B$ \\ and $G_a : B \to C$} &
    $\begin{array}{@{}l@{}}
      \swapFun_{ABC}(G,b)\\
      \substFun_{ABC}(\constFun_{AB}(b),G)
    \end{array}$
    with
    $\begin{aligned}
      G : A &\to     (B \to C)\\
          a &\mapsto G_a
    \end{aligned}$\\
    \hline
    \rule{0pt}{2em}
    $\begin{aligned}
      F : A &\to     B\\
          a &\mapsto G_a(a)
    \end{aligned}$ &
    for $G_a : A \to B$ &
    $\begin{array}{@{}l@{}}
      \dupFun_{AB}(G)\\
      \substFun_{AAB}(\idFun_A,G)
    \end{array}$
    with
    $\begin{aligned}
      G : A &\to     (A \to B)\\
          a &\mapsto G_a
    \end{aligned}$
  \end{longtable}

  Equivalences between all alternatives can be obtained directly from the axioms
  $\nm{rightId},\nm{leftId},\ldots$ without invoking the functor universe.
  However, this alone would not be sufficient to ensure that there is an equivalence
  between the corresponding final results of the functoriality algorithm.
\end{remark}

\subsection{Functorial meta-relations}

\section{Singletons}
\label{sec:singletons}

\section{Products}
\label{sec:products}

\section{Equivalences}
\label{sec:equivalences}

\section{Properties and relations}
\label{sec:properties}
\label{sec:relations}

\section{Dependent functors}

\section{Dependent products}

\end{document}
