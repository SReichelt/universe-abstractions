\documentclass[a4paper]{article}

\usepackage[margin=2.5cm]{geometry}
\usepackage[parfill]{parskip}
\usepackage[utf8]{inputenc}
\usepackage{titlesec}
\usepackage{xcolor}
\usepackage{hyperref}

\usepackage{amsmath,amssymb,amsfonts,amsthm}

\title{Dependent Combinators}
\author{Sebastian Reichelt}

\titleformat{\paragraph}{\bfseries}{\theparagraph}{}{}

\setlength{\skip\footins}{0.5cm}

\theoremstyle{definition}
\newtheorem{definition}{Definition}[section]
\newtheorem{theorem}[definition]{Theorem}
\newtheorem{proposition}[definition]{Proposition}
\newtheorem{lemma}[definition]{Lemma}
\newtheorem{corollary}[definition]{Corollary}
\newtheorem{conjecture}[definition]{Conjecture}
\theoremstyle{remark}
\newtheorem*{remark}{Remark}
\newtheorem*{remarks}{Remarks}
\newtheorem*{example}{Example}

\newcommand{\defn}{\emph}

\newcommand{\eqDef}{\overset{\text{def}}{=}}

\renewcommand{\phi}{\varphi}
\renewcommand{\equiv}{\simeq}
\renewcommand{\emptyset}{\varnothing}
\newcommand{\iso}{\cong}
\newcommand{\equivto}{\overset{\equiv}{\rightsquigarrow}}

\newcommand{\ZF}{\mathsf{ZF}}

\newcommand{\A}{\mathcal{A}}
\newcommand{\B}{\mathcal{B}}
\newcommand{\C}{\mathcal{C}}
\newcommand{\D}{\mathcal{D}}
\newcommand{\U}{\mathcal{U}}
\newcommand{\V}{\mathcal{V}}
\newcommand{\W}{\mathcal{W}}

\newcommand{\nm}{\mathsf}

\newcommand{\universe}{\nm}
\newcommand{\Unit}{\universe{Unit}}
\newcommand{\Bool}{\universe{Bool}}
\newcommand{\Prop}{\universe{Prop}}
\newcommand{\Set}{\universe{Set}}
\newcommand{\Cat}{\universe{Cat}}

\newcommand{\Hom}{\mathrm{Hom}}
\newcommand{\Iso}{\mathrm{Iso}}

\newcommand{\id}{\nm{id}}
\newcommand{\apply}{\nm{apply}}
\newcommand{\intro}{\nm{intro}}
\newcommand{\elim}{\nm{elim}}
\newcommand{\refl}{\nm{refl}}
\newcommand{\symm}{\nm{symm}}
\newcommand{\trans}{\nm{trans}}
\newcommand{\congrArg}{\nm{congrArg}}
\newcommand{\congrFun}{\nm{congrFun}}
\newcommand{\fndef}[1]{\nm{def}_{#1}}

\newcommand{\combinator}{\nm}
\newcommand{\idFun}{\combinator{I}}
\newcommand{\revAppFun}{\combinator{T}}
\newcommand{\constFun}{\combinator{K}}
\newcommand{\compFun}{\combinator{B'}}
\newcommand{\revCompFun}{\combinator{B}}
\newcommand{\swapFun}{\combinator{C}}
\newcommand{\dupFun}{\combinator{W}}
\newcommand{\revSelfAppFun}{\combinator{O}}
\newcommand{\substFun}{\combinator{S'}}
\newcommand{\revSubstFun}{\combinator{S}}
\newcommand{\constPi}{\combinator{Kd}}
\newcommand{\appPi}{\combinator{Ad}}
\newcommand{\compFunPi}{\combinator{Bd}}

\newcommand{\rightId}{\nm{rightId}}
\newcommand{\leftId}{\nm{leftId}}
\newcommand{\assoc}{\nm{assoc}}

\newcommand{\swapRevApp}{\nm{swapT}}
\newcommand{\swapSwap}{\nm{swapSwap}}
\newcommand{\swapCompFun}{\nm{swapB'}}
\newcommand{\swapRevCompFun}{\nm{swapB}}
\newcommand{\compAssoc}{\nm{assoc}}
\newcommand{\rightConst}{\nm{rightConst}}
\newcommand{\leftConst}{\nm{leftConst}}
\newcommand{\dupSwap}{\nm{dupC}}
\newcommand{\dupConst}{\nm{dupK}}
\newcommand{\rightDup}{\nm{rightDup}}
\newcommand{\substDup}{\nm{leftDup}}
\newcommand{\dupDup}{\nm{dupW}}
\newcommand{\substAltDef}{\nm{dupLeftC}}
\newcommand{\substAssoc}{\nm{assocS}}

\newcommand{\idFunExt}{\nm{extI}}
\newcommand{\revAppFunExt}{\nm{extT}}
\newcommand{\compFunExt}{\nm{extB'}}
\newcommand{\constFunExt}{\nm{extK}}
\newcommand{\dupFunExt}{\nm{extW}}

\newcommand{\Lean}{\texttt}
\newcommand{\redundant}{\textcolor{gray}}

\begin{document}

\maketitle

\section{Introduction}

This project, which is still work in progress, is an attempt to define structures axiomatically in
such a way that these axioms have models in all conceivable collections of higher categories,
including bicategories and hopefully any of the different notions of $\infty$-categories (and also,
of course, 1-categories and sets).
Although somewhat inspired by Homotopy Type Theory (HoTT) \cite{hottbook}, the entire theory is
compatible with standard (and, to a large extend, weaker) foundations, including Zermelo-Fraenkel
($\ZF$) set theory and the dependent type theory implemented by the Lean 4 \cite{lean4} theorem
prover. A primary objective is to automate abstract but mathematically trivial proofs in Lean, and
this document is intended to serve as documentation of that formalization.

The guiding principle (which somewhat matches categorical thinking outside of the HoTT realm) is to
never assume equality in any axiom, as that equality becomes a more general equivalence at the next
higher level. In particular, this means that 1-categories are not the right starting point for a
completely generic formalization. Instead, the axioms have a type-theoretic flavor, with categories
being just one of many models.

We do not, however, assume any knowledge of type theory -- except in the following overview, which
is specifically written with readers coming from a type theory background in mind. The rest of this
document, as well as the formalization in Lean 4, are self-contained except for the descriptions of
category-theoretic models of the axioms.

\section{Overview}

This section aims to give just enough of an overview so that readers with some knowledge of type
theory can get an idea what to expect and not to expect.

One particular thing \emph{not} to expect is a definition of a specific type theory, at least not in
the usual sense, even though this documentation uses type-theoretic notation throughout. Instead,
all definitions should be read primarily as notation for axioms defined in set theory -- though
it does make sense to treat each model of these axioms as a model of a type theory. This axiomatic
approach allows us to avoid any discussion (and Lean formalization) of syntax, which would just
distract from the goal of automating proofs \emph{outside} of the type theory.

The axioms are organized in ``layers.'' Although a strong connection between layers and
type-theoretical universe levels (or homotopy levels in HoTT) exists, there is one important
difference: Each model tends to satisfy the axioms of \emph{all} layers. If a model has too little
internal structure for a particular layer, then it satisfies the axioms of that layer trivially
instead of not at all.

For example, the axioms of the first layer can loosely be interpreted as defining
\defn{propositions}, but those same axioms are \emph{also} part of all other layers. One can think
of layer $n$ as defining constraints governing each span of $n$ successive universe levels.

\begin{itemize}
  \item \textbf{Layer 1} starts from a very simple definition of a Tarski-style \defn{universe}:
  an index set $I$ (of \defn{types}) together with a function that assigns a set of \defn{instances}
  (each written as $a : A$) to each type $A \in I.$ (Note that this is a self-contained
  set-theoretic definition and not tied to any particular type theory.) Moreover:
  \begin{itemize}
    \item For any two types $A$ and $B$, we assume the existence of a type $[A \to B]$ such that
    given an instance $F : A \to B$ and an instance $a : A$ we can obtain an instance $F(a) : B.$
    We call $F$ a \defn{functor} due to its additional structure introduced in layers 2 and 3.
    (In layer 1, we might call it an `implication' instead.)
    \item We axiomatically assert the existence of certain functors, which in this layer correspond
    to logical axioms, and in general are combinators \cite{combinators}. Thus in order to build
    functors, we can use the well-known algorithm to convert arbitrary lambda terms into
    applications of combinators.
    \item We can optionally introduce further types such as \defn{units}, \defn{counits},
    \defn{products}, and \defn{coproducts}. Functorial introduction and elimination rules enable
    their integration into the above algorithm.
    \item Lastly, for types $A$ and $B$ we introduce an \defn{equivalence type}
    $[A \leftrightarrow B]$ and assert the existence of certain instances of this type, in
    particular corresponding to reflexivity, symmetry, and transitivity. We also assert that there
    is an instance of $[[A \leftrightarrow B] \to [A \to B]].$
    (Importantly, equivalence is \emph{not} simply defined to be the product of two functors, as
    that definition would not be reusable at higher layers.)
  \end{itemize}
  \item A \textbf{layer 2} universe is layer 1 universe $\U$ along with another layer 1 universe
  $\V,$ which can be thought of as the next-lower-level universe. We introduce equivalences between
  instances as well as a limited notion of dependent types.
  \begin{itemize}
    \item For every type $A : \U$ and instances $a,b : A,$ we demand the existence of a type
    $[a \equiv b]$ in $\V$ that satisfies the axioms of a (functorial $\V$-valued) equivalence
    relation.
    \item For each of the concrete functors defined in layer 1, we introduce an axiom that specifies
    the behavior of the functor up to equivalence.
    \item For two functors $F : A \to B$ and $G : A \to C,$ we introduce a type
    $[F \overset{\equiv}{\rightsquigarrow} G] : \V$ with axioms that let us interpret it as
    \[\prod_{a,b : A}\:[F(a) \equiv F(b) \to G(a) \equiv G(b)].\]
    Similarly to the combinators we use to build concrete functors, we define
    \defn{dependent combinators} to convert lambda terms into instances of this type. The
    combinators also imply
    \[\congrArg_F : \prod_{a,b : A}\:[a \equiv b \to F(a) \equiv F(b)].\]
    \item Additional dependent combinators guarantee \defn{extensionality} of functors with respect
    to equivalence. This allows us to extend the $\Pi$ notation by simply defining
    \[\prod_{a : A}\:[F(a) \equiv G(a)]\]
    as notation for $[F \equiv G],$ as we can automatically translate any lambda term implementing
    this $\Pi$ type to a proof of $F \equiv G.$ (In particular, this automates some tedious category
    theory proofs.)
    \item Similarly to $\Pi$ types in $\V,$ we introduce $\Sigma$ types in $\U,$ i.e.\ subtypes.
    \item All types are required to respect instance equivalences in specific ways, so that the
    translation of lambda terms into dependent combinators also works for terms involving those
    types. In particular, the two functors of an equivalence are required to be inverse to each
    other up to instance equivalence.
    \item We can axiomatically embed $\V$ in $\U,$ obtaining a notion of \defn{truncation}.
    \item If $\V$ appears as a type in $\U,$ so that we have types of the form $[A \to \V]$ and
    $[A \to A \to \V],$ we can define a more generic notion of dependent types that implies the
    above axioms. Moreover, we can demand that instance equivalences of $\V$ are the same as type
    equivalences. (This tends to fail if $\V$ is too large.)
  \end{itemize}
  \item \textbf{Layer 3} is derived from layer 2 by strengthening the universe $\V$ to a layer 2
  universe. This means that we have equivalences of equivalences in a third (layer 1) universe $\W,$
  suggesting an interpretation of the higher-level equivalences as isomorphisms.
  \begin{itemize}
    \item Types and instance equivalences in layer 3 are required to form a (weak) groupoid,
    functors are required to be groupoid functors, and so on.
    \item In particular, equivalences are required to be adjoint equivalences.
    \item Optionally, we may add a concept of morphisms between types of $\U,$ by stating axioms
    that ensure compatibility with instance equivalences when those are regarded as isomorphisms.
    \item If $\V$ is a type in $\U,$ we can demand that morphisms of $\V$ are the same as functors.
  \end{itemize}
\end{itemize}

\section{Layer 1}

This section introduces the `propositional' axioms that all layers are based on, without any
prerequisites.

\subsection{Universes}
\label{sec:layer1-universes}

\begin{definition}
  \label{def:layer1-universe}
  We define a \defn{layer 1 universe} to be a pair $(I, (S_A)_{A \in I})$ of
  \begin{itemize}
    \item an index set (or more general collection) $I,$ the members of which we call \defn{types},
    and
    \item a family $(S_A)_{A \in I}$ of sets indexed by a type. We call the members of $S_A$ the
    \defn{instances} of $A.$
  \end{itemize}
\end{definition}

Given a universe $\U = (I, (S_A)_{A \in I}),$ we write
\begin{itemize}
  \item ``$A : \U$'' for ``$A \in I$'' and
  \item ``$a : A$'' for ``$a \in S_A.$''
\end{itemize}

(So this definition allows us to use type-theoretic notation without actually defining a type
theory with concrete syntax.)

There are obviously a lot of examples of universes as defined above. Some concrete universes that we
will refer to throughout the rest of the document are:

\begin{itemize}
  \item A universe $\Unit$ with a single type $\ast$ that has a single instance $\star : \ast.$
  \item A universe $\Bool$ with an empty type $\bot$ and a type $\top$ with a single instance
  $\star : \top.$\footnote{In the Lean formalization of this theory, we have two different universes
  corresponding to the Lean types \Lean{Bool} and \Lean{Prop}. Both of them map to $\Bool$ in this
  document.}
  \item A universe $\Set$ of sets, such that $S_A := A$ for each set $A.$\footnote{It is possible
  to use a universe of setoids in place of the universe of sets. In the Lean formalization, this
  universe is the simplest nontrivial universe in a certain sense.}
  \item Universes of algebraic data structures, such that $I$ is the collection of groups, vector
  spaces over a given field, etc., and $S_A$ is the carrier set of the structure $A.$
  \item A universe $\Cat$ of categories, such that $S_A$ is the set of objects of the category $A.$
  \item Generalization from $\Cat$ to higher categories should be straightforward. (Ideally, we
  would like to \emph{define} $\infty$-categories based on the universe axioms, but this is still
  work in progress.)
\end{itemize}

\subsection{Functors}

\begin{definition}
  We say that a layer 1 universe $\U = (I, (S_A)_{A \in I})$ has \defn{functors} if for all types
  $A,B : \U$ we have a \defn{functor type} $[A \to B] : \U$ and a function
  $\apply_{AB} : S_{[A \to B]} \times S_A \to S_B.$
\end{definition}

Given instances $F : A \to B$ and $a : A,$ we write ``$F(a)$'' for ``$\apply_{AB}(F,a).$''
As usual, we let ``$\to$'' be right-associative, i.e.\ the notation ``$A \to B \to C$'' stands for
``$A \to [B \to C].$'' We call such a functor $F$ a \defn{bifunctor} and write ``$F(a,b)$'' for
``$F(a)(b).$'' (In section \ref{sec:layer1-products}, we will identify $A \to B \to C$ with
$A \times B \to C,$ where $A \times B$ is a product type.)

Most of the universes defined above have functor types:
\begin{itemize}
  \item In $\Unit,$ there is obviously only one choice $[\ast \to \ast] := \ast$ and
  $\star(\star) := \star.$
  \item In $\Bool,$ functors can be defined as logical implication:
  \[[A \to B] := \begin{cases}
    \bot & \text{if } A = \top \text{ and } B = \bot,\\
    \top & \text{otherwise}
  \end{cases}\]
  and $\star(\star) := \star.$
  \item Functions in $\Set$ are functors.
  \item In $\Cat,$ we can let $[A \to B]$ be the functor category from $A$ to $B.$
  \item Universes of some algebraic structures have functors, depending on whether morphisms are
  again instances of the same structure. For example, this is the case for commutative semigroups
  and for vector spaces.
\end{itemize}

\begin{definition}
  For a set $S$ and a universe $\U = (I,\dots)$ with functors, we define a
  \defn{$\U$-valued prerelation} on $S$ to be a function $(\prec) : S \times S \to I,$ which we
  write in infix form.
  We say that $(\prec)$ is
  \begin{itemize}
    \item \defn{reflexive} if for every $a \in S$ we have an instance $\id_a : a \prec a,$
    \item \defn{symmetric} if for every $a,b \in S$ we have a functor
    $(-)^{-1} : a \prec b \to b \prec a,$ and
    \item \defn{transitive} if for every $a,b,c \in S$ we have a bifunctor
    $(- \circ -) : a \prec b \to b \prec c \to a \prec c.$
  \end{itemize}
\end{definition}

(See also \cite{hottbook}, section 2.1.)

Note in particular that the functor arrow $(\to)$ is a $\U$-valued prerelation on $I.$

\begin{definition}
  \label{def:layer1-combinators}
  A layer 1 universe $\U$ with functors has \defn{linear logic} if for all types $A,B,C : \U$ there
  are functors
  \begin{itemize}
    \item $\idFun_A : A \to A,$
    \item $\revAppFun_{AB} : A \to [A \to B] \to B,$ and
    \item $\compFun_{ABC} : [A \to B] \to [B \to C] \to [A \to C].$
  \end{itemize}
  It has \defn{affine logic} if there is additionally a functor
  \begin{itemize}
    \item $\constFun_{AB} : B \to [A \to B],$
  \end{itemize}
  and \defn{full logic} if there is additionally\footnote{We ignore \defn{relevant logic} because it
  would add more complexity to definition \ref{def:layer2-dependent-combinators}.}
  \begin{itemize}
    \item $\dupFun_{AB} : [A \to A \to B] \to [A \to B].$
  \end{itemize}

  We generally omit type subscripts when the types are fully determined by arguments. Moreover,
  if $\U$ has (at least) linear logic, then the prerelation $(\to)$ is reflexive and transitive with
  \begin{itemize}
    \item $\id_A = \idFun_A$ and
    \item $G \circ F = \compFun(F,G)$ for $F : A \to B$ and $G : B \to C.$
  \end{itemize}
\end{definition}

\begin{proposition}
From the functors defined above we can derive the following functors with given types.
  \begin{itemize}
    \item $\swapFun_{ABC} := \compFun_{B[[B{\to}C]{\to}C][A{\to}C]}(\revAppFun_{BC}) \circ \compFun_{A[B{\to}C]C} :
                             [A \to B \to C] \to [B \to A \to C],$
    \item $\revCompFun_{ABC} := \swapFun(\compFun_{ABC}) :
                                [B \to C] \to [A \to B] \to [A \to C],$
    and, assuming full logic,
    \item $\revSelfAppFun_{AB} := \dupFun_{[A{\to}B]B} \circ \compFun_{[A{\to}B]AB} :
                                  [[A \to B] \to A] \to [A \to B] \to B,$
    \item $\begin{aligned}[t]
      \substFun_{ABC} &:= \revCompFun_{[A{\to}B{\to}C][A{\to}A{\to}C][A{\to}C]}(\dupFun_{AC}) \circ
                          \revCompFun_{A[B{\to}C][A{\to}C]} \circ
                          \compFun_{ABC} :\\
                      &\quad\ \ [A \to B] \to [A \to B \to C] \to [A \to C],
    \end{aligned}$
    \item $\revSubstFun_{ABC} := \swapFun(\substFun_{ABC}) :
                                 [A \to B \to C] \to [A \to B] \to [A \to C].$
  \end{itemize}
\end{proposition}
\vspace{1ex}

From the choice of names, it may be obvious that these functors can be interpreted as combinators
\cite{combinators} in a simply-typed lambda calculus \cite{simply-typed-lambda-calculus}. Therefore,
we can construct functors using the standard algorithm to convert lambda terms to combinator
applications.

Instead of lambda notation, we will write such constructed functors as \defn{functor descriptions}
\begin{align*}
  F : A_1 \to \dots \to A_n &\to     B\\
      (a_1,\dots,a_n)       &\mapsto t_{a_1 \dots a_n},
\end{align*}
and state the algorithm to construct a functor corresponding to a given description in a slightly
informal way as follows.

\begin{proposition}
  \label{prp:layer1-functoriality}
  Let $\U$ be a universe with (at least) linear logic, $A_1,\dots,A_n$ and $B$ be types in $\U,$
  and $(t_{a_1 \dots a_n})_{a_k : A_k}$ be a family of instances $t_{a_1 \dots a_n} : B$
  that are one of the following:
  \begin{itemize}
    \item a constant independent of all $a_k,$
    \item one of the variables $a_k,$ or
    \item a functor application $G_{a_1 \dots a_n}(b_{a_1 \dots a_n})$ such that both
    $G_{a_1 \dots a_n}$ and $b_{a_1 \dots a_n}$ recursively follow the same rules.\footnote{More
    formally, the rules generate a family $(T_B)_{B : \U}$ of sets $T_B \subseteq F_B,$ where
    $F_B$ is the set of functions from $S_{A_1} \times \dots \times S_{A_n}$ to $S_B,$ and $S_A$ is
    the set of instances of $A$ for each $A : \U.$}
  \end{itemize}
  Then we can construct a functor
  \begin{align*}
    F : A_1 \to \dots \to A_n &\to     B\\
        (a_1,\dots,a_n)       &\mapsto t_{a_1 \dots a_n}
  \end{align*}
  if the following additional constraints are satisfied.
  \begin{itemize}
    \item If $\U$ has linear but not affine logic, each variable $a_k$ must occur exactly once in
    $t_{a_1 \dots a_n}.$
    \item If $\U$ has affine but not full logic, each variable $a_k$ must occur at most once
    in $t_{a_1 \dots a_n}.$
  \end{itemize}
\end{proposition}

\begin{proof}
  First, note that the case $n > 1$ can be reduced to two recursive applications of the same
  algorithm with lower $n$ as follows. For constant $a_1 : A_1,$ construct the functor
  \begin{align*}
    F_{a_1} : A_2 \to \dots \to A_n &\to     B\\
              (a_2,\dots,a_n)       &\mapsto t_{a_1 \dots a_n}.
  \end{align*}
  Then the result is given by
  \begin{align*}
    F : A_1 &\to     [A_2 \to \dots \to A_n \to B]\\
        a_1 &\mapsto F_{a_1}.
  \end{align*}

  Due to this reduction, we can limit ourselves to the simple case
  \begin{align*}
    F : A &\to     B\\
        a &\mapsto t_a
  \end{align*}
  and perform a (non-unique) case split on $t_a$ according to table \ref{tbl:layer1-functoriality}.

  \begin{table}
    \centering
    \begin{tabular}{ll|l}
      Definition & & Functor\\
      \hline\hline
      \rule{0pt}{2em}
      $\begin{aligned}
        F : A &\to     B\\
            a &\mapsto b
      \end{aligned}$ &
      \parbox{3cm}{for $b : B$ (constant \\ with respect to $a$)} &
      $\constFun_{AB}(b)$\\
      \hline
      \rule{0pt}{2em}
      $\begin{aligned}
        F : A &\to     A\\
            a &\mapsto a
      \end{aligned}$ & &
      $\idFun_A$\\
      \hline
      \rule{0pt}{2em}
      $\begin{aligned}
        F : A &\to     B\\
            a &\mapsto G(a)
      \end{aligned}$ &
      for $G : A \to B$ &
      $G$\\
      \hline
      \rule{0pt}{2em}
      $\begin{aligned}
        F : A &\to     C\\
            a &\mapsto G(b_a)
      \end{aligned}$ &
      \parbox{3cm}{for $b_a : B$ \\ and $G : B \to C$} &
      $\compFun(H,G)$ with
      $\begin{aligned}
        H : A &\to     B\\
            a &\mapsto b_a
      \end{aligned}$\\
      \hline
      \rule{0pt}{2em}
      $\begin{aligned}
        F : [B \to C] &\to     C\\
            G         &\mapsto G(b)
      \end{aligned}$ &
      for $b : B$ &
      $\revAppFun_{BC}(b)$\\
      \hline
      \rule{0pt}{2em}
      $\begin{aligned}
        F : A &\to     C\\
            a &\mapsto G_a(b)
      \end{aligned}$ &
      \parbox{3cm}{for $b : B$ \\ and $G_a : B \to C$} &
      $\swapFun(G,b)$ with
      $\begin{aligned}
        G : A &\to     [B \to C]\\
            a &\mapsto G_a
      \end{aligned}$\\
      \hline
      \rule{0pt}{2em}
      $\begin{aligned}
        F : A &\to     B\\
            a &\mapsto G_a(a)
      \end{aligned}$ &
      for $G_a : A \to B$ &
      $\dupFun(G)$ with
      $\begin{aligned}
        G : A &\to     [A \to B]\\
            a &\mapsto G_a
      \end{aligned}$\\
      \hline
      \rule{0pt}{2em}
      $\begin{aligned}
        F : [B \to C] &\to     C\\
            G         &\mapsto G(b_G)
      \end{aligned}$ &
      for $b_G : B$ &
      $\revSelfAppFun(H)$ with
      $\begin{aligned}
        H : [B \to C] &\to     B\\
            G         &\mapsto b_G
      \end{aligned}$\\
      \hline
      \rule{0pt}{3em}
      $\begin{aligned}
        F : A &\to     C\\
            a &\mapsto G_a(b_a)
      \end{aligned}$ &
      \parbox{3cm}{for $b_a : B$ \\ and $G_a : B \to C$} &
      \parbox{6cm}{
        $\substFun(H,G)$ with \\[1ex]
        $\begin{aligned}
          H : A &\to     B\\
              a &\mapsto b_a
        \end{aligned}$
        and
        $\begin{aligned}
          G : A &\to     [B \to C]\\
              a &\mapsto G_a
        \end{aligned}$
      }
    \end{tabular}
    \caption{Conversion of functor description to combinator application}
    \label{tbl:layer1-functoriality}
  \end{table}

  In all cases except the first two, $t_a$ is a functor application. In fact, the last case is the
  most general possible functor application, and in a universe with full logic, all other cases of
  functor applications may be regarded as mere optimizations.
\end{proof}

\begin{remarks}
  Note that a functor application with multiple arguments $F(a_1,\dots,a_n)$ is really an
  application of the functor $F(a_1,\dots,a_{n-1})$ to the argument $a_n,$ and must be treated as
  such.

  The algorithm may produce terms of the form $\swapFun(\compFun,\ldots)$ or
  $\swapFun(\substFun,\ldots).$ By the definitions of $\revCompFun$ and $\revSubstFun,$
  these can be replaced with $\revCompFun(\ldots)$ and $\revSubstFun(\ldots),$
  respectively.
\end{remarks}

\begin{proposition}
  \label{prp:layer1-combinator-def}
  The algorithm applied to each of the following functor descriptions outputs exactly the given
  combinator, when taking the topmost possible alternative in case of nondeterminism.
  \begin{itemize}
    \item $\begin{aligned}[t]
      \constFun_{AB} : B \to A &\to     B\\
                       (b,a)   &\mapsto b
    \end{aligned}$
    \item $\begin{aligned}[t]
      \idFun_A : A &\to     A\\
                 a &\mapsto a
    \end{aligned}$
    \item $\begin{aligned}[t]
      \compFun_{ABC} : [A \to B] \to [B \to C] \to A &\to     C\\
                       (F,G,a)                       &\mapsto G(F(a))
    \end{aligned}$
    \item $\begin{aligned}[t]
      \revCompFun_{ABC} : [B \to C] \to [A \to B] \to A &\to     C\\
                          (G,F,a)                       &\mapsto G(F(a))
    \end{aligned}$
    \item $\begin{aligned}[t]
      \revAppFun_{AB} : A \to [A \to B] &\to     B\\
                        (a,F)           &\mapsto F(a)
    \end{aligned}$
    \item $\begin{aligned}[t]
      \swapFun_{ABC} : [A \to B \to C] \to B \to A &\to     C\\
                       (F,b,a)                     &\mapsto F(a,b)
    \end{aligned}$
    \item $\begin{aligned}[t]
      \dupFun_{AB} : [A \to A \to B] \to A &\to     B\\
                     (F,a)                 &\mapsto F(a,a)
    \end{aligned}$
    \item $\begin{aligned}[t]
      \revSelfAppFun_{AB} : [[A \to B] \to A] \to [A \to B] &\to     B\\
                            (F,G)                           &\mapsto G(F(G))
    \end{aligned}$
    \item $\begin{aligned}[t]
      \substFun_{ABC} : [A \to B] \to [A \to B \to C] \to A &\to     C\\
                        (F,G,a)                             &\mapsto G(a,F(a))
    \end{aligned}$
    \item $\begin{aligned}[t]
      \revSubstFun_{ABC} : [A \to B \to C] \to [A \to B] \to A &\to     C\\
                           (G,F,a)                             &\mapsto G(a,F(a))
    \end{aligned}$
  \end{itemize}
\end{proposition}

\begin{remark}
  Note, however, that this does \emph{not} imply that functors map instances exactly as specified
  by the definition, as in layer 1 there are no axioms about specific instances. However, for the
  concrete universes listed earlier this in fact the case.
\end{remark}

\begin{proposition}
  \label{prp:layer1-logic}
  Of the universes listed earlier,
  \begin{itemize}
    \item $\Unit,$ $\Bool,$ $\Set,$ and $\Cat$ have full logic, and
    \item the universes of commutative semigroups and of vector spaces have linear logic.
  \end{itemize}
\end{proposition}

\begin{proof}
  The required functors can be defined easily. A strategy that works for all universes is to
  algorithmically construct the combinators from smaller functors as shown in proposition
  \ref{prp:layer1-combinator-def}. E.g.\ for $\revAppFun,$ the algorithm specifies that one first
  needs to construct a functor
  \begin{align*}
    T_a : [A \to B] &\to     B\\
          F         &\mapsto F(a)
  \end{align*}
  and then to construct
  \begin{align*}
    \revAppFun_{AB} : A &\to     [[A \to B] \to B]\\
                      a &\mapsto T_a,
  \end{align*}
  both of which are either trivial or simple depending on the universe.

  Alternatively, a universe has full logic if its types form a Cartesian closed category, and linear
  logic if its types form a closed monoidal category. (Note that the converse may not necessarily
  hold, due to the lack of axioms on instances.)
\end{proof}

\begin{remark}
  Specifically for $\Cat,$ this means that we can construct functors between categories by simply
  stating their functor description, without having to verify manually that this definition is
  functorial. Similarly for linear maps between vector spaces, though in that case we need to
  ensure that each variable is used exactly once.
\end{remark}

\subsection{Singletons}
\label{sec:layer1-singletons}

\begin{definition}
  A layer 1 universe $\U$ with functors may have a designated \defn{unit type} $\top : \U$ which
  must satisfy the following two axioms.
  \begin{itemize}
    \item It must have an instance $\star : \top.$
    \item For each type $A : \U$ there is a functor $\elim_A : A \to \top \to A$ (which matches
    $\constFun_{\top A}$ but should exist even in the absence of affine logic, encoding the idea
    that a unit instance contains no data).
  \end{itemize}
\end{definition}

\begin{definition}
  Similarly, $\U$ may have an \defn{counit type} $\bot : \U,$ which must satisfy the
  following axiom.
  \begin{itemize}
    \item For each type $A : \U$ there is a functor $\elim_A : \bot \to A.$
  \end{itemize}
\end{definition}

\begin{remark}
  Note that the above axioms do not imply that $\top$ has a single instance or that $\bot$ has no
  instances. In particular the type $\ast : \Unit$ is both a unit type and a counit type.
\end{remark}

\subsection{Products and coproducts}
\label{sec:layer1-products}

\begin{definition}
  A layer 1 universe $\U$ with functors has \defn{products} if for all $A,B : \U$ we have a
  \defn{product type} $[A \times B] : \U$ with the following functors.
  \begin{itemize}
    \item $\intro_{AB} : A \to B \to A \times B$ (taking the symbol ``$\times$'' to bind more
    strongly than ``$\to$'')
    \item $\elim_{ABC} : [A \to B \to C] \to [A \times B \to C]$ for each $C : \U$
  \end{itemize}

  For $a : A$ and $b : B$ we define their pair as
  \[(a,b) := \intro(a,b)\]
  and extend the functor description notation such that
  \begin{align*}
    F : A \times B &\to     C\\
        (a,b)      &\mapsto t_{ab}
  \end{align*}
  is defined as $F := \elim(F')$ with
  \begin{align*}
    F' : A \to B &\to     C\\
         (a,b)   &\mapsto t_{ab}.
  \end{align*}
\end{definition}

\begin{remark}
  This implies that projections from $[A \times B]$ to $A$ or $B$ are only available under the
  assumption of (at least) affine logic, except if the other side is $\top.$ However, linear logic
  gives us, for example,
  \begin{align*}
    \nm{comm}_{AB} : A \times B &\to     B \times A\\
                     (a,b)      &\mapsto (b,a)
  \end{align*}
  and
  \begin{align*}
    \nm{assoc}_{ABC} : [A \times B] \times C &\to     A \times [B \times C]\\
                       ((a,b),c)             &\mapsto (a,(b,c)).
  \end{align*}
\end{remark}

\begin{definition}
  $\U$ has \defn{coproducts} if for all $A,B : \U$ we have a \defn{coproduct type}
  $[A \oplus B] : \U$ with the following functors.
  \begin{itemize}
    \item $\nm{lintro}_{AB} : A \to A \oplus B$
    \item $\nm{rintro}_{AB} : B \to A \oplus B$
    \item $\elim_{ABC} : [A \to C] \to [B \to C] \to [A \oplus B \to C]$ for each $C : \U$
  \end{itemize}
\end{definition}

\subsection{Equivalences}
\label{sec:layer1-equivalences}

\begin{definition}
  A layer 1 universe $\U$ with functors has \defn{equivalences} if for all $A,B : \U$ we have an
  \defn{equivalence type} $[A \leftrightarrow B] : \U$ such that $(\leftrightarrow)$ is a reflexive
  symmetric transitive prerelation and for $A,B : \U$ we have a functor
  \begin{itemize}
    \item $\elim_{AB} : [A \leftrightarrow B] \to [A \to B].$
  \end{itemize}
\end{definition}

Due to the lack of an introduction functor, these axioms do not enable the construction of concrete
equivalences between two different types. This is intentional because the correct notion of
equivalence depends on the universe $\U,$ for example equivalences in $\Set$ are bijections, whereas
equivalences in $\Cat$ are (adjoint) equivalences of categories.

Therefore, we assert the existence of concrete equivalences axiomatically. In all of these cases,
the corresponding functors in both directions can already be proven to exist regardless of the
equivalence. In other words, the axioms do not add any logical strength, but they establish a
concept of equivalence that can be referenced at higher layers.

\begin{definition}
  We say that $\U$ has \defn{standard equivalences} if it has (at least) linear logic and instances
  of the following types exist, for all $A,B,C : \U.$
  \begin{itemize}
    \item $[A \to B \to C] \leftrightarrow [B \to A \to C]$
    \item $[\top \to A] \leftrightarrow A$ if $\U$ has a unit type
    \item $[A \to \bot] \leftrightarrow [A \leftrightarrow \bot]$ if $\U$ has a counit type
    \item $[A \times B \to C] \leftrightarrow [A \to B \to C]$ if $\U$ has products
    \item $A \times B \leftrightarrow B \times A$ if $\U$ has products
    \item $[A \times B] \times C \leftrightarrow A \times [B \times C]$ if $\U$ has products
    \item $[A \to B \times C] \leftrightarrow [A \to B] \times [A \to C]$ if $\U$ has products
    and full logic
    \item $\top \times A \leftrightarrow A$ if $\U$ has products and a unit type
    \item $\bot \times A \leftrightarrow \bot$ if $\U$ has products and a counit type
    \item $A \oplus B \leftrightarrow B \oplus A$ if $\U$ has coproducts
    \item $[A \oplus B] \oplus C \leftrightarrow A \oplus [B \oplus C]$ if $\U$ has coproducts
    \item $[A \oplus B \to C] \leftrightarrow [A \to C] \times [B \to C]$ if $\U$ has coproducts
    and products
    \item $\bot \oplus A \leftrightarrow A$ if $\U$ has coproducts and a counit type
    \item $[A \leftrightarrow B] \leftrightarrow [B \leftrightarrow A]$
    \item functors of the form $[A \leftrightarrow B] \to [\Phi(A) \leftrightarrow \Phi(B)],$ where
    $\Phi : \U \to \U$ is a function that constructs types
  \end{itemize}
\end{definition}

\subsection{Functor universe}
\label{sec:layer1-functor-universe}

In this section, we construct a universe $\U^A$ that can be understood as ``$\U$ within a context
containing an instance of $A : \U.$'' If we regard the types of $\U$ as propositions (via the
Curry-Howard correspondence \cite{curry-howard}), then in order to prove an implication
$[A \to B] : \U,$ we can instead ``prove $B$ under the assumption $A$'' by constructing an instance
of $B^A : \U^A.$

We therefore want $\U^A$ to inherit all structure of $\U,$ and whether or not it does so is a good
coherence test of the axioms, particularly at higher layers.

\begin{definition}
  Let $\U$ be a universe with full logic, and $A : \U$ be a type. We define the
  \defn{functor universe} $\U^A$ as follows.
  \begin{itemize}
    \item The types of $\U^A,$ written as $B^A : \U^A$ for $B : \U,$ are in 1:1 correspondence with
    the types of $\U.$ (In fact, we can technically define the set of types of $\U^A$ to be exactly
    the set of types of $\U,$ but for clarity we will assume the types to be distinct in this
    description.)
    \item The set of instances of $B^A : \U^A$ is the set of instances of $[A \to B] : \U.$
  \end{itemize}
\end{definition}

Then $\U^A$ has, among others, the following instances.
\begin{itemize}
  \item $\idFun_A : A^A$ (which acts as the ``instance of $A$ in the context''), and
  \item for each $B : \U$ and $b : B,$ an ``embedded'' instance $b^A := \constFun_{AB}(b) : B^A.$
\end{itemize}

\begin{proposition}
  For $B,C : \U,$ the type $[B \to C]^A : \U^A$ is a functor type from $B^A : \U^A$ to $C^A : \U^A.$
  Under this definition of functors, $\U^A$ has full logic.
\end{proposition}
\vspace{-1ex}
\begin{proof}
  For $G : [B \to C]^A$ and $F : B^A$ we can define $G(F) := \substFun_{ABC}(F,G) : C^A.$

  For a functor $H : B \to C,$ its embedded instance $H^A : [B \to C]^A$ is a functor from $B^A$ to
  $C^A.$ Thus, we can define the five primitive combinators by embedding the corresponding
  combinators of $\U$:
  $\idFun_{B^A} := (\idFun_B)^A,$ $\revAppFun_{B^A C^A} := (\revAppFun_{BC})^A$, $\ldots$
\end{proof}

\begin{proposition}
  For types $B,C : U,$ consider the functor description
  \begin{align*}
    G : B &\to     C\\
        b &\mapsto t_b.
  \end{align*}
  If $t_b$ follows the constraints of proposition \ref{prp:layer1-functoriality}. we can lift the
  family $(t_b)_{b : B}$ to a family $(T_F)_{F : B^A}$ of instances of $C^A$ such that the functor
  description
  \begin{align*}
    {G^A}' : B^A &\to     C^A\\
             F   &\mapsto T_F
  \end{align*}
  also follows these constraints.
\end{proposition}

\begin{proof}
  We recursively define $T_F$ based on the three possibilities for $t_b.$
  \begin{itemize}
    \item If $t_b$ is a constant $c$ independent of $b,$ we set $T_F := c^A.$
    \item If $t_b = b,$ we set $T_F := F.$
    \item If $t_b$ is a functor application, we set $T_F$ to the application of
    the lifted functor to the lifted term. \qedhere
  \end{itemize}
\end{proof}

We can use this proposition to obtain an alternative proof of proposition
\ref{prp:layer1-functoriality} as follows (assuming full logic).
Given a family $(t_a)_{a : A}$ of terms $t_a : B$ that satisfy the constraints of proposition
\ref{prp:layer1-functoriality}, we may lift $(t_a)$ to a family $(T_F)_{F : A^A}$ of terms
$T_F : B^A.$ Then $T_{\idFun_A} : B^A$ is a functor from $A$ to $B.$

\begin{proposition}
  If $\U$ has a unit type $\top : \U,$ then $\top^A$ is a unit type of $\U^A.$
\end{proposition}
\vspace{-1ex}
\begin{proof}
  Obviously we have $\star^A : \top^A.$ For $B : \U$ we can take $\elim_{B^A} := (\elim_B)^A.$
\end{proof}

Counits, products, coproducts, and equivalences embed analogously.

\section{Layer 2}

\subsection{Universes}

\begin{definition}
  A \defn{layer 2 universe} $(\U,\V,(\equiv_A)_{A : \U})$ is a triple of
  \begin{itemize}
    \item a layer 1 universe $\U,$
    \item a layer 1 universe $\V$ with (at least) linear logic, and
    \item for each type $A : \U,$ a reflexive symmetric transitive $\V$-valued prerelation
    $(\equiv_A)$ on the set of instances of $A.$ We call $[a \equiv b] : \V$ the type of
    \defn{instance equivalences} between $a : A$ and $b : A.$
  \end{itemize}

  We will often write ``$\U$'' instead of ``$(\U,\V,(\equiv_A)_{A : \U})$,'' leaving $\V$ and
  $(\equiv_A)_{A : \U}$ implicit.
\end{definition}

In particular, we will consider the following layer 2 universes.
\begin{itemize}
  \item $(\Unit, \Unit, (\star,\star) \mapsto \ast)$
  \item $(\Bool, \Unit, (\star,\star) \mapsto \ast)$
  \item $(\Set, \Bool, (=))$
  \item $(\Cat, \Set, \Iso),$ where $\Iso(a,b)$ is the set of isomorphisms from $a$ to $b.$
  (This requires special consideration of size-related issues, which we will ignore here because
  they are adequately addressed by the formalization in Lean.)
\end{itemize}

\subsection{Functors}

\begin{definition}
  A layer 2 universe $(\U,\V,(\equiv))$ has \defn{functors} if $\U$ has functors (as a layer 1
  universe) and additionally for all types $A,B,C : \U$ and functors $F : A \to B$ and $G : A \to C$
  we have a type $[F \equivto G] : \V$ that satisfies the following axioms.
  \begin{itemize}
    \item For $a,b : A$ there is a functor\\
    $\elim_{FGab} : [F \equivto G] \to [F(a) \equiv F(b) \to G(a) \equiv G(b)].$
    \item For $A,B : \U$ and $F,G : A \to B$ there is a functor\\
    $\nm{eq}_{FG} : F \equiv G \to [F \equivto G].$
    \item For $A,B,C,D : \U,$ $F : A \to B,$ $G : A \to C,$ and $H : A \to D$ there is a bifunctor\\
    $\nm{trans}_{FGH} : [F \equivto G] \to [G \equivto H] \to [F \equivto H].$
    \item If $\U$ has (at least) linear logic, then for $A,B,C : \U,$ $F : A \to B,$ and
    $G : B \to C$ there is an instance\\
    $\nm{rBr}_{FG} : [F \equivto G \circ F].$
    \item If $\U$ has (at least) linear logic, then for $A,B,C,D : \U,$ $F : A \to B,$
    $G : A \to C,$ and $H : B \to D$ there is a functor\\
    $\nm{rBl}_{FGH} : [H \circ F \equivto G] \to [F \equivto G].$
    \item If $\U$ has (at least) affine logic, then for $A,B,C : \U,$ $F : A \to B,$ and $c : C$
    there is an instance\\
    $\nm{rK}_{Fc} : F \equivto \constFun_{AC}(c).$
    % TODO: Is it possible to reduce the next two to something with dupFun?
    \item If $\U$ has full logic, then for $A,B,C,D : \U,$ $F : A \to B,$ $G : A \to C,$ and
    $H : A \to C \to D$ there is a functor\\
    $\nm{rSr}_{FGH} : [F \equivto G] \to [F \equivto H] \to [F \equivto \substFun(G,H)].$
    \item If $\U$ has full logic, then for $A,B,C,D : \U,$ $F : A \to B,$ $G : A \to C,$ and
    $H : A \to B \to D$ there is a functor\\
    $\nm{rSl}_{FGH} : [\substFun(F,H) \equivto G] \to [H \equivto G] \to [F \equivto G].$
  \end{itemize}
\end{definition}

In the concrete universes we consider, this type is implemented as follows.
\begin{itemize}
  \item In $\Unit$ and $\Bool,$ obviously $[F \equivto G] := \ast : \Unit$ for all functors $F$ and
  $G.$
  \item For $A,B,C : \Set,$ $F : A \to B,$ and $G : A \to C,$ we define
  \[[F \equivto G] := \begin{cases}
    \top & \text{if for all } a,b \in A, F(a) = F(b) \text{ implies } G(a) = G(b),\\
    \bot & \text{otherwise.}
  \end{cases}\]
  \item For $A,B,C : \Cat,$ $F : A \to B,$ and $G : A \to C,$ we define $[F \equivto G]$ to be a
  set of \defn{functorial maps}, which are collections of functions from $\Hom(F(a),F(b))$ to
  $\Hom(G(a),G(b)),$ depending on two objects $a,b : A,$ that respect identity and composition of
  morphisms.
\end{itemize}

\begin{definition}
  In a layer 2 universe $\U$ with functors, when we say that a functor $F$ matches a given functor
  description
  \begin{align*}
    F : A_1 \to \dots \to A_n &\to     B\\
        (a_1,\dots,a_n)       &\mapsto t_{a_1 \dots a_n},
  \end{align*}
  we require that there is an instance equivalence
  $\fndef{F}(a_1,\dots,a_n) : F(a_1,\dots,a_n) \equiv t_{a_1 \dots a_n}$ for $a_k : A_k.$
\end{definition}

For $\U$ to have linear, affine, or full logic, we will require each of the five primitive
combinators to match their description as given in proposition \ref{prp:layer1-combinator-def}.

\begin{remark}
  By $\nm{rBr},$ this implies that that if $\U$ has (at least) linear logic, then
  \begin{itemize}
    \item for all $A,B : \U,$ $F : A \to B,$ and $a,b : A$ there is an instance\\
    $\congrArg_{Fab} : a \equiv b \to F(a) \equiv F(b),$ and
    \item for all $A,B : \U,$ $F,G : A \to B,$ and $a : A$ there is an instance\\
    $\congrFun_{FGa} : F \equiv G \to F(a) \equiv G(a).$
  \end{itemize}
\end{remark}

\begin{proposition}
  If the five primitive combinators match their functor descriptions as defined above, then the
  algorithm to construct a functor for a given functor description, as described in the proof of
  proposition \ref{prp:layer1-functoriality}, always produces functors that match the functor
  description that is given as input to the algorithm.
\end{proposition}
\vspace{-1ex}
\begin{proof}
  The required equivalence can be constructed by composing, in each step of the algorithm, the
  definition of the combinator used in that step with the definitions of its arguments, using
  $\congrArg.$
\end{proof}

\subsection{Dependent functors}

\begin{definition}
  \label{def:layer2-dependent-functor-type}
  For a layer 2 universe layer 2 universe $(\U,\V,(\equiv))$ with functors, we introduce a
  \defn{dependent functor type} notation
  \[\left[\prod_{a_k : A_k} p_{a_1 \dots a_n}\right] : \V\]
  for some specific terms $p_{a_1 \dots a_n} : \V$ with $A_1,\dots,A_n : \U.$
  (A more generic definition that only works for some universes is given later, starting in section
  \ref{sec:layer2-universe-type}.)

  The common requirement on this notation is that given a term
  \[f : \prod_{a_k : A_k} p_{a_1 \dots a_n}\]
  and concrete $a_k : A_k,$ we have a \defn{dependent functor application}
  \[f(a_1,\dots,a_n) : p_{a_1 \dots a_n}.\]

  \begin{itemize}
    \item For the case that $p_{a_1 \dots a_n} : \V$ is an instance equivalence, we simply define
    \[\left[\prod_{a_k : A_k} [s_{a_1 \dots a_n} \equiv t_{a_1 \dots a_n}]\right] := [F \equiv G]\]
    with $B : \U$ such that $s_{a_1 \dots a_n},t_{a_1 \dots a_n} : B$ and
    \begin{align*}
      F : A_1 \to \dots \to A_n &\to     B\\
          (a_1,\dots,a_n)       &\mapsto s_{a_1 \dots a_n},\\[1ex]
      G : A_1 \to \dots \to A_n &\to     B\\
          (a_1,\dots,a_n)       &\mapsto t_{a_1 \dots a_n}.
    \end{align*}
    To avoid nondeterminism within the specification of the dependent functor type, the functors $F$
    and $G$ must be obtained from $(s_{a_1 \dots a_n})$ and $(t_{a_1 \dots a_n})$ by always taking
    the topmost possible alternative within table \ref{tbl:layer1-functoriality}.

    Note that the algorithm to construct $F$ and $G$ ensures that
    \[\left[\prod_{a_1 : A_1, \dots, a_n : A_n} [s_{a_1 \dots a_n} \equiv t_{a_1 \dots a_n}]\right] \eqDef
      \left[\prod_{a_1 : A_1, \dots, a_{n-1} : A_{n-1}} \left[\prod_{a_n : A_n} [s_{a_1 \dots a_n} \equiv t_{a_1 \dots a_n}]\right]\right]\]
    if $n > 1.$
    \item A specific case of implication is given by
    \[\left[\prod_{a,b : A} [s_a \equiv s_b \to t_a \equiv t_b]\right] := [F \equivto G]\]
    with $B,C : \U$ such that $s_a : B$ and $t_a : C$ and
    \begin{align*}
      F : A &\to     B\\
          a &\mapsto s_a,\\[1ex]
      G : A &\to     C\\
          a &\mapsto t_a.
    \end{align*}
    \item Quantification over products can be defined by distribution:
    \[\left[\prod_{a_k : A_k} [p_{a_1 \dots a_n} \times q_{a_1 \dots a_n}]\right] :=
      \left[\left[\prod_{a_k : A_k} p_{a_1 \dots a_n}\right] \times
            \left[\prod_{a_k : A_k} q_{a_1 \dots a_n}\right]\right]\]
    \item Quantification over a unit type is defined to yield a unit type:
    \[\left[\prod_{a_k : A_k} \top\right] := \top\]
  \end{itemize}
\end{definition}

As for functors, we wish to define instances $f : \prod_{a_k : A_k} p_{a_1 \dots a_n}$ via
\defn{dependent functor descriptions}
\begin{align*}
  f : A_1 \to \dots \to A_n &\to     p_{a_1 \dots a_n}\\
      (a_1,\dots,a_n)       &\mapsto t_{a_1 \dots a_n},
\end{align*}
so we will list specific conditions that enable this.

\begin{definition}
  \label{def:layer2-dependent-combinators}
  A layer 2 universe $(\U,\V,(\equiv))$ with functors has \defn{linear logic} if the layer 1
  universe $\U$ has linear logic, the definitions of $\idFun,$ $\revAppFun,$ and $\compFun$ match
  their descriptions as given in proposition \ref{prp:layer1-combinator-def}, and for all types
  $A,B,C,D : \U$ there are instances
  \begin{itemize}
    \item $\begin{aligned}[t]
      \rightId_{AB} &: \prod_{F : A \to B, a : A} [F(\idFun(a)) \equiv F(a)]\\
                    &\ \ \eqDef \prod_{F : A \to B} [F \circ \idFun_A \equiv F]\\
                    &\ \ \eqDef [\compFun_{AAB}(\idFun_A) \equiv \idFun_{A{\to}B}],
    \end{aligned}$
    \item $\begin{aligned}[t]
      \leftId_{AB} &: \prod_{F : A \to B, a : A} [\idFun(F(a)) \equiv F(a)]\\
                   &\ \ \eqDef \prod_{F : A \to B} [\idFun_B \circ F \equiv F]\\
                   &\ \ \eqDef [\revCompFun_{ABB}(\idFun_B) \equiv \idFun_{A{\to}B}],
    \end{aligned}$
    \item $\begin{aligned}[t]
      \swapRevApp_{AB} &: \prod_{F : A \to B, a : A} [\revAppFun(a,F) \equiv F(a)]\\
                       &\ \ \eqDef \prod_{F : A \to B} [\swapFun(\revAppFun_{AB},F) \equiv F]\\
                       &\ \ \eqDef [\swapFun(\revAppFun_{AB}) \equiv \idFun_{A{\to}B}],
    \end{aligned}$
    \item $\begin{aligned}[t]
      \swapCompFun_{ABC} &: \prod_{F : A \to B, a : A, G : B \to C} [(G \circ F)(a) \equiv G(F(a))]\\
                         &\ \ \eqDef \prod_{F : A \to B} [\swapFun(\compFun_{ABC}(F)) \equiv \revAppFun_{BC} \circ F]\\
                         &\ \ \eqDef [\swapFun_{[B{\to}C]AC} \circ \compFun_{ABC} \equiv \revCompFun_{AB[[B{\to}C]{\to}C]}(\revAppFun_{BC})],
    \end{aligned}$
    \item $\begin{aligned}[t]
      \swapRevCompFun_{ABC} &: \prod_{G : B \to C, a : A, F : A \to B} [(G \circ F)(a) \equiv G(F(a))]\\
                            &\ \ \eqDef \prod_{G : B \to C} [\swapFun(\revCompFun_{ABC}(G)) \equiv \revCompFun_{[A{\to}B]BC}(G) \circ \revAppFun_{AB}]\\
                            &\ \ \eqDef [\swapFun_{[A{\to}B]AC} \circ \revCompFun_{ABC} \equiv \revCompFun_{A[[A{\to}B]{\to}B][[A{\to}B]{\to}C]}(\revAppFun_{AB}) \circ \revCompFun_{[A{\to}B]BC}],
    \end{aligned}$
    \item $\begin{aligned}[t]
      \compAssoc_{ABCD} &: \prod_{F : A \to B, G : B \to C, H : C \to D, a : A} [(H \circ G)(F(a)) \equiv H(G(F(a)))]\\
                        &\ \ \eqDef \prod_{F : A \to B, G : B \to C, H : C \to D} [(H \circ G) \circ F \equiv H \circ (G \circ F)]\\
                        &\ \ \eqDef [\compFun_{[B{\to}C][[C{\to}D]{\to}[B{\to}D]][[C{\to}D]{\to}[A{\to}D]]}(\compFun_{BCD}) \circ \revCompFun_{[C{\to}D][B{\to}D][A{\to}D]} \circ \compFun_{ABD} \equiv\\
                        &\ \ \quad\ \ \,\revCompFun_{[B{\to}C][A{\to}C][[C{\to}D]{\to}[A{\to}D]]}(\compFun_{ACD}) \circ \compFun_{ABC}].
    \end{aligned}$
  \end{itemize}

  $(\U,\V,(\equiv))$ has \defn{affine logic} it it has linear logic, the layer 1 universes $\U$ and
  $\V$ have affine logic, the definition of $\constFun$ matches its description as given in
  proposition \ref{prp:layer1-combinator-def}, and for all types $A,B,C : \U$ there are instances
  \begin{itemize}
    \item $\begin{aligned}[t]
      \rightConst_{ABC} &: \prod_{b : B, G : B \to C, a : A} [G(\constFun(b,a)) \equiv G(b)]\\
                        &\ \ \eqDef \prod_{b : B, G : B \to C} [G \circ \constFun_{AB}(b) \equiv \constFun_{AC}(G(b))]\\
                        &\ \ \eqDef [\compFun_{ABC} \circ \constFun_{AB} \equiv \revCompFun_{[B{\to}C]C[A{\to}C]}(\constFun_{AC}) \circ \revAppFun_{BC}],
    \end{aligned}$
    \item $\begin{aligned}[t]
      \leftConst_{ABC} &: \prod_{F : A \to B, c : C, a : A} [\constFun(c,F(a)) \equiv c]\\
                       &\ \ \eqDef \prod_{F : A \to B, c : C} [\constFun_{BC}(c) \circ F \equiv \constFun_{AC}(c)]\\
                       &\ \ \eqDef [\compFun_{C[B{\to}C][A{\to}C]}(\constFun_{BC}) \circ \compFun_{ABC} \equiv \constFun_{[A{\to}B][C{\to}A{\to}C]}(\constFun_{AC})].
    \end{aligned}$
  \end{itemize}

  $(\U,\V,(\equiv))$ has \defn{full logic} if it has affine logic, the layer 1 universes $\U$ and
  $\V$ have full logic, the definition of $\dupFun$ matches its description as given in proposition
  \ref{prp:layer1-combinator-def}, and for all types $A,B,C : \U$ there are instances
  \begin{itemize}
    \item $\begin{aligned}[t]
      \dupSwap_{AB} &: \prod_{F : A \to A \to B, a : A} [\swapFun(F,a,a) \equiv F(a,a)]\\
                    &\ \ \eqDef \prod_{F : A \to A \to B} [\dupFun(\swapFun(F)) \equiv \dupFun(F)]\\
                    &\ \ \eqDef [\dupFun_{AB} \circ \swapFun_{AAB} \equiv \dupFun_{AB}],
    \end{aligned}$
    \item $\begin{aligned}[t]
      \dupConst_{AB} &: \prod_{F : A \to B, a : A} [\constFun(F,a,a) \equiv F(a)]\\
                     &\ \ \eqDef \prod_{F : A \to B} [\dupFun(\constFun_{A[A{\to}B]}(F)) \equiv F]\\
                     &\ \ \eqDef [\dupFun_{AB} \circ \constFun_{A[A{\to}B]} \equiv \idFun_{A{\to}B}].
    \end{aligned}$
    \item $\begin{aligned}[t]
      \rightDup_{ABC} &: \prod_{F : A \to A \to B, G : B \to C, a : A} [G(\dupFun(F,a)) \equiv G(F(a,a))]\\
                      &\ \ \eqDef \prod_{F : A \to A \to B, G : B \to C} [G \circ \dupFun(F) \equiv \dupFun(\revCompFun_{ABC}(G) \circ F)]\\
                      &\ \ \eqDef [\compFun_{ABC} \circ \dupFun_{AB} \equiv\\
                      &\ \ \quad\ \ \,\revCompFun_{[B{\to}C][A{\to}A{\to}C][A{\to}C]}(\dupFun_{AC}) \circ \compFun_{[B{\to}C][[A{\to}B]{\to}[A{\to}C]][A{\to}A{\to}C]}(\revCompFun_{ABC})\,\circ\\
                      &\hspace{30em}\compFun_{A[A{\to}B][A{\to}C]}],
    \end{aligned}$
    \item $\begin{aligned}[t]
      \substDup_{ABC} &: \prod_{F : A \to B, G : A \to B \to B \to C, a : A} [\dupFun(G(a),F(a)) \equiv G(a,F(a),F(a))]\\
                      &\ \ \eqDef \prod_{F : A \to B, G : A \to B \to B \to C} [\substFun(F,\dupFun_{BC} \circ G) \equiv \substFun(F,\substFun(F,G))]\\
                      &\ \ \eqDef [\compFun_{[A{\to}B{\to}B{\to}C][A{\to}B{\to}C][A{\to}C]}(\revCompFun_{A[B{\to}B{\to}C][B{\to}C]}(\dupFun_{BC})) \circ \substFun_{ABC} \equiv\\
                      &\ \ \quad\ \ \,\substFun(\substFun_{ABC},\compFun_{[A{\to}B{\to}B{\to}C][A{\to}B{\to}C][A{\to}C]} \circ \substFun_{AB[B{\to}C]})].
    \end{aligned}$
  \end{itemize}
\end{definition}

\begin{proposition}
  \label{prp:layer2-derived-dependent-functors}
  The following instances can be derived from the axioms above, for $A,B,C,D : \U,$ assuming linear
  or full logic as appropriate.
  \begin{enumerate}
    \item $\begin{aligned}[t]
      \swapSwap_{ABC} &: \prod_{F : A \to B \to C} [\swapFun(\swapFun(F)) \equiv F]
    \end{aligned}$
    \item $\begin{aligned}[t]
      \substAltDef_{ABC} &: \prod_{F : A \to B, G : A \to B \to C} [\dupFun(\swapFun(G) \circ F) \equiv \substFun(F,G)]
    \end{aligned}$
    \item $\begin{aligned}[t]
      \dupDup_{AB} &: \prod_{F : A \to A \to A \to B} [\dupFun(\dupFun_{AB} \circ F) \equiv \dupFun(\dupFun(F))]
    \end{aligned}$
    \item $\begin{aligned}[t]
      \substAssoc_{ABCD} &: \prod_{F : A \to B, G : A \to B \to C, H : A \to C \to D} [\substFun(F,\substFun(G,\revCompFun_{BCD} \circ H)) \equiv \substFun(\substFun(F,G),H)]
    \end{aligned}$
  \end{enumerate}
\end{proposition}

\begin{proof}
  We will only list the axioms that play a central role in the proof.
  \begin{enumerate}
    \item By $\swapRevApp.$
    \item By $\dupSwap$ and $\swapSwap.$
    \item By $\substDup.$
    \item By $\compAssoc$ and $\dupDup.$
  \end{enumerate}
  Full proofs (and further equivalences) are contained in the Lean formalization.
\end{proof}

\vspace{1ex}
\begin{lemma}
  \label{lem:layer2-functoriality-step-equiv}
  When a single step in the functoriality algorithm is nondeterministic according to table
  \ref{tbl:layer1-functoriality}, the resulting functors are equivalent.
\end{lemma}
\vspace{-1ex}
\begin{proof}
  The specific cases of nondeterminism are listed in table \ref{tbl:layer2-functoriality-step-equiv}.
  (Grayed-out alternatives are redundant because they can also be regarded as alternatives of
  other alternatives.)

  \begin{table}
    \centering
    \begin{tabular}{ll|l}
      Case & & Alternatives\\
      \hline\hline
      \rule{0pt}{3em}
      $\begin{aligned}
        F : A &\to     C\\
            a &\mapsto G(b)
      \end{aligned}$ &
      \parbox{3cm}{for $b : B$ \\ and $G : B \to C$} &
      $\begin{array}{@{}l@{}}
        \constFun_{AC}(G(b))\\
        \compFun(\constFun_{AB}(b),G)\\
        \swapFun(\constFun_{A[B{\to}C]}(G),b)\\
        \redundant{\substFun(\constFun_{AB}(b),\constFun_{A[B{\to}C]}(G))}
      \end{array}$\\
      \hline
      \rule{0pt}{3em}
      $\begin{aligned}
        F : A &\to     B\\
            a &\mapsto G(a)
      \end{aligned}$ &
      for $G : A \to B$ &
      $\begin{array}{@{}l@{}}
        G\\
        \compFun(\idFun_A,G)\\
        \redundant{\dupFun(\constFun_{A[A{\to}B]}(G))}\\
        \redundant{\substFun(\idFun_A,\constFun_{A[A{\to}B]}(G))}
      \end{array}$\\
      \hline
      \rule{0pt}{2em}
      $\begin{aligned}
        F : A &\to     C\\
            a &\mapsto G(b_a)
      \end{aligned}$ &
      \parbox{3cm}{for $G : B \to C$ \\ and $b_a : B$} &
      $\begin{array}{@{}l@{}}
        \compFun(H,G)\\
        \substFun(H,\constFun_{B[B{\to}C]}(G))
      \end{array}$
      with
      $\begin{aligned}
        H : A &\to     B\\
            a &\mapsto b_a
      \end{aligned}$\\
      \hline
      \rule{0pt}{3em}
      $\begin{aligned}
        F : [B \to C] &\to     C\\
            G         &\mapsto G(b)
      \end{aligned}$ &
      for $b : B$ &
      $\begin{array}{@{}l@{}}
        \revAppFun_{BC}(b)\\
        \swapFun(\idFun_{B{\to}C},b)\\
        \redundant{\revSelfAppFun(\constFun_{[B{\to}C]B}(b))}\\
        \redundant{\substFun(\constFun_{[B{\to}C]B}(b),\idFun_{B{\to}C})}
      \end{array}$\\
      \hline
      \rule{0pt}{2em}
      $\begin{aligned}
        F : A &\to     C\\
            a &\mapsto G_a(b)
      \end{aligned}$ &
      \parbox{3cm}{for $b : B$ \\ and $G_a : B \to C$} &
      $\begin{array}{@{}l@{}}
        \swapFun(G,b)\\
        \substFun(\constFun_{AB}(b),G)
      \end{array}$
      with
      $\begin{aligned}
        G : A &\to     [B \to C]\\
            a &\mapsto G_a
      \end{aligned}$\\
      \hline
      \rule{0pt}{2em}
      $\begin{aligned}
        F : A &\to     B\\
            a &\mapsto G_a(a)
      \end{aligned}$ &
      for $G_a : A \to B$ &
      $\begin{array}{@{}l@{}}
        \dupFun(G)\\
        \substFun(\idFun_A,G)
      \end{array}$
      with
      $\begin{aligned}
        G : A &\to     [A \to B]\\
            a &\mapsto G_a
      \end{aligned}$\\
      \hline
      \rule{0pt}{2em}
      $\begin{aligned}
        F : [B \to C] &\to     C\\
            G         &\mapsto G(b_G)
      \end{aligned}$ &
      for $b_G : B$ &
      $\begin{array}{@{}l@{}}
        \revSelfAppFun(H)\\
        \substFun(H,\idFun_{B{\to}C})
      \end{array}$
      with
      $\begin{aligned}
        H : [B \to C] &\to     B\\
            G         &\mapsto b_G
      \end{aligned}$\\
    \end{tabular}
    \caption{Nondeterminism in the algorithm to construct functors}
    \label{tbl:layer2-functoriality-step-equiv}
  \end{table}

  Equivalences between the alternatives given on the right are easily derived from the axioms and
  proposition \ref{prp:layer2-derived-dependent-functors}.
\end{proof}

\begin{lemma}
  \label{lem:layer2-combinator-def-equiv}
  For each primitive combinator $F$ with functor description
  \begin{align*}
    F : A_1 \to \dots \to A_n &\to     B\\
        (a_1,\dots,a_n)       &\mapsto t_{a_1 \dots a_n},
  \end{align*}
  there is an instance of
  \[\prod_{x : X}\:[F({a_1}_x,\dots,{a_n}_x) \equiv t_{{a_1}_x \dots {a_n}_x}]\]
  if the families $({a_1}_x)_{x : X},\dots,({a_n}_x)_{x : X}$ satisfy the conditions of proposition
  \ref{prp:layer1-functoriality}.

  Specifically, we have the following dependent functors for types $X,A,B,C : \U$ and suitable
  families $(a_x),(b_x),(F_x),(G_x)$.
  \begin{enumerate}
    \item $\begin{aligned}[t]
      \idFunExt_{XA}((a_x)_{x : X}) &: \prod_{x : X}\:[\idFun(a_x) \equiv a_x]
    \end{aligned}$\\
    with $a_x : A,$
    \item $\begin{aligned}[t]
      \revAppFunExt_{XAB}((a_x)_{x : X},(F_x)_{x : X}) &: \prod_{x : X}\:[\revAppFun(a_x,F_x) \equiv F_x(a_x)]
    \end{aligned}$\\
    with $a_x : A$ and $F_x : A \to B,$
    \item $\begin{aligned}[t]
      \compFunExt_{XABC}((F_x)_{x : X},(G_x)_{x : X},(a_x)_{x : X}) &: \prod_{x : X}\:[\compFun(F_x,G_x,a_x) \equiv G_x(F_x(a_x))]
    \end{aligned}$\\
    with $F_x : A \to B,$ $G_x : B \to C,$ and $a_x :A,$
    \item $\begin{aligned}[t]
      \constFunExt_{XAB}((b_x)_{x : X},(a_x)_{x : X}) &: \prod_{x : X}\:[\constFun(b_x,a_x) \equiv b_x]
    \end{aligned}$\\
    with $b_x : B$ and $a_x : A,$
    \item $\begin{aligned}[t]
      \dupFunExt_{XAB}((F_x)_{x : X},(a_x)_{x : X}) &: \prod_{x : X}\:[\dupFun(F_x,a_x) \equiv F_x(a_x,a_x)]
    \end{aligned}$\\
    with $F_x : A \to A \to B$ and $a_x : A.$
  \end{enumerate}
\end{lemma}

\begin{proof}
  According to definition \ref{def:layer2-dependent-functor-type}, for each of the five combinators
  we need to execute the algorithm to construct a functor on both sides of each equivalence, taking
  the first alternative whenever the algorithm is nondeterministic, and construct an equivalence
  between the resulting functors. Thus, the required constructions are different depending on
  whether each argument is constant, exactly $x,$ or dependent on $x.$

  Fortunately, lemma \ref{lem:layer2-functoriality-step-equiv} allows us to combine these three
  cases whenever the requirement on linear vs.\ affine vs.\ full logic is the same.\footnote{A
  practical implementation of the algorithm may still want to handle each case individually to
  produce shorter terms.}
  In order to handle linear and affine logic, we need to provide an individual construction for each
  case where exactly one of the arguments depends on $x,$ but we do not need to consider the case
  where an argument is exactly $x.$ In situations where full logic is required, we can limit
  ourselves to the case that \emph{all} arguments depend on $x.$

  This leaves us with the following list, where dependent arguments are written as ``$(t_x)$'' and
  constant arguments are written as ``$(t).$''\\
  In each case, for a family $(t_x)_{x : X}$ of terms $t_x : T$, let $\phi(t_x) : X \to T$ denote
  the functor constructed for $(t_x)_{x : X}$ under the assumption that $t_x$ depends on $x.$

  \begin{enumerate}
    \item $\idFunExt_{XA}((a_x)) := \leftId(\phi(a_x)).$
    \item $\revAppFunExt_{XAB}((a),(F_x)) := \id_{\swapFun(\phi(F_x),a)}.$\\
    $\revAppFunExt_{XAB}((a_x),(F)) : \swapFun(\revAppFun_{AB} \circ \phi(a_x),F) \equiv F \circ \phi(a_x)$
    is constructed from $\swapRevApp.$\\
    $\revAppFunExt_{XAB}((a_x),(F_x)) : \substFun(\phi(F_x),\revAppFun_{AB} \circ \phi(a_x)) \equiv \substFun(\phi(a_x),\phi(F_x))$
    is constructed from $\substAltDef.$
    \item $\compFunExt_{XABC}((F),(G),(a_x)) := \compAssoc(\phi(a_x),F,G).$\\
    $\compFunExt_{XABC}((F),(G_x),(a)) : \swapFun(\compFun_{ABC}(F) \circ \phi(G_x),a) \equiv \swapFun(\phi(G_x),F(a))$
    is constructed from $\swapCompFun.$\\
    $\compFunExt_{XABC}((F_x),(G),(a)) : \swapFun(\revCompFun_{ABC}(G) \circ \phi(F_x),a) \equiv G \circ \swapFun(\phi(F_x),a)$
    is constructed from $\swapRevCompFun.$\\
    $\compFunExt_{XABC}((F_x),(G_x),(a_x)) := \substAssoc(\phi(a_x),\phi(F_x),\phi(G_x)).$
    \item $\constFunExt_{XAB}((b),(a_x)) := \leftConst(\phi(a_x),b).$\\
    $\constFunExt_{XAB}((b_x),(a)) : \swapFun(\constFun_{AB} \circ \phi(b_x),a) \equiv \phi(b_x)$
    is constructed from $\rightConst.$\\
    $\constFunExt_{XAB}((b_x),(a_x)) : \substFun(\phi(a_x),\constFun_{AB} \circ \phi(b_x)) \equiv \phi(b_x)$
    is constructed from $\rightConst,$ $\leftConst,$ $\dupConst,$ and $\dupSwap.$
    \item $\dupFunExt$ requires full logic; therefore everything can be handled by\\
    $\dupFunExt_{XAB}((F_x),(a_x)) := \substDup(\phi(a_x),\phi(F_x)).$
    \qedhere
  \end{enumerate}
\end{proof}

\begin{theorem}
  \label{thm:layer2-extensionality}
  Let $\U$ be a layer 2 universe with functors, $A_1,\dots,A_n,B$ be types in $\U,$
  $(s_{a_1 \dots a_n})_{a_k : A_k}$ and $(t_{a_1 \dots a_n})_{a_k : A_k}$ be families
  of instances $s_{a_1 \dots a_n},t_{a_1 \dots a_n} \in B,$ and
  $(e_{a_1 \dots a_n})_{a_k : A_k}$ be a family of instance equivalences
  $e_{a_1 \dots a_n} : s_{a_1 \dots a_n} \equiv t_{a_1 \dots a_n}$ that are one of the
  following:
  \begin{itemize}
    \item a constant independent of all $a_k$ (implying that $s_{a_1 \dots a_n}$ and
    $t_{a_1 \dots a_n}$ are also constant),
    \item $\id_{s_{a_1 \dots a_n}}$ (implying $s_{a_1 \dots a_n} = t_{a_1 \dots a_n}$),
    \item $f_{a_1 \dots a_n}^{-1}$ for an equivalence $f_{a_1 \dots a_n}$ that recursively
    follows the same rules,
    \item $g_{a_1 \dots a_n} \circ f_{a_1 \dots a_n}$ for equivalences $f_{a_1 \dots a_n}$
    and $g_{a_1 \dots a_n}$ that recursively follow the same rules,
    % TODO: Change to a more general \equivto instance.
    \item $\congrArg_{G_{a_1 \dots a_n}}(f_{a_1 \dots a_n})$ for a functor $G_{a_1 \dots a_n}$ that
    satisfies the constraint of proposition \ref{prp:layer1-functoriality} and an equivalence
    $f_{a_1 \dots a_n}$ that recursively satisfies the constraints of this theorem,
    \item an application of a functor definition
    $\fndef{G}(x_{a_1 \dots a_n},x'_{a_1 \dots a_n},\dots),$ where $G$ is a primitive combinator,
    such that $x_{a_1 \dots a_n},x'_{a_1 \dots a_n},\dots$ satisfy the constraint of proposition
    \ref{prp:layer1-functoriality}.
  \end{itemize}

  Then we can construct an equivalence $e : F_s \equiv F_t$ between functors $F_s$ and $F_t,$
  corresponding to the dependent functor description
  \begin{align*}
    e : A_1 \to \dots \to A_n &\to     e_{a_1 \dots a_n}\\
        (a_1,\dots,a_n)       &\mapsto s_{a_1 \dots a_n} \equiv t_{a_1 \dots a_n}.
  \end{align*}
\end{theorem}

\begin{proof}
  As in the proof of proposition \ref{prp:layer1-functoriality}, we can restrict ourselves to the
  case that $n = 1,$ i.e.\ we replace $a_1,\dots,a_n$ with a single $a : A.$
  We recursively construct $e$ based on the different cases.
  \begin{itemize}
    \item If $s_a$ and $t_a$ are constants $s,t : B,$ then $F_s = \constFun_{AB}(s)$ and
    $F_t = \constFun_{AB}(t).$ Therefore, if $e_a$ is also a constant $f,$ we can set
    $e := \congrArg_{\constFun_{AB}}(f).$
    \item If $e_a = \id_{s_a}$ implying $s_a = t_a$ for all $a,$ then $F_s = F_t,$ and we
    can set $e := \id_{A{\to}B}.$
    \item If $e_a = f_a^{-1},$ set $e := f^{-1},$ where $f$ is the equivalence obtained
    recursively for $f_a.$
    \item Likewise for $e_a = g_a \circ f_a.$
    \item If $e_a = \congrArg_{G_a,u_a,v_a}(f_a)$ for an appropriate functor
    $G_a : C \to B$ and equivalence $f_a : u_a \equiv v_a$ with $u_a,v_a : C,$ then the
    definitions of $F_s$ and $F_t$ depend on whether $G_a,$ $u_a,$ and $v_a$ are constant,
    exactly $a,$ or dependent on $a.$ If $u_a$ and $v_a$ differ in this respect, first
    obtain equivalent alternatives for $F_s$ and $F_t$ according to lemma
    \ref{lem:layer2-functoriality-step-equiv}, so that $F_s$ and $F_t$ are both applications of
    the same combinator $H.$ Then $e$ is obtained by applying $H$ to $f$ in the
    appropriate way, where $f$ is the equivalence obtained recursively for $f_a.$
    \item If $e_a = \fndef{G}(x_a,x'_a,\dots)$ for a combinator $G,$ set $e$ according
    to lemma \ref{lem:layer2-combinator-def-equiv}.
    \qedhere
  \end{itemize}
\end{proof}

\begin{corollary}
  \label{cor:layer2-extensionality}
  If $F,G : A_1 \to \dots \to A_n \to B$ are functors and
  $(e_{a_1 \dots a_n})_{a_k : A_k}$ is a family of instance equivalences
  $e_{a_1 \dots a_n} : F(a_1,\dots,a_n) \equiv G(a_1,\dots,a_n)$ that satisfy the
  constraints of theorem \ref{thm:layer2-extensionality}, then we have an equivalence
  $e : F \equiv G.$
\end{corollary}

\begin{corollary}
  Whenever the algorithm to construct a functor is nondeterministic, there are equivalences
  between the resulting functors.\\
  (This strengthens lemma \ref{lem:layer2-functoriality-step-equiv}, as an equivalence in
  each step does not necessarily translate to an equivalence in the final result.)
\end{corollary}

\begin{remarks}
  The triviality or nontriviality of this theorem depends on the amount of structure
  that instance equivalences of $\U$ have. In the universe of categories, it is
  already quite significant: If for two functors $F,G : \C \to \D$ we can derive a
  family $(e_c)_{c \in \C}$ of isomorphisms from the axioms given in the previous
  sections, then $F$ and $G$ are naturally isomorphic.

  As with the functoriality algorithm, the usefulness of the theorem increases with
  additional structure that we define on universes. When such structure is equipped
  with appropriate functors and equivalences, theorem \ref{thm:layer2-extensionality}
  applies to that additional structure as well.

  The result can also be interpreted as an extensionality theorem in lambda calculus,
  or more specifically in combinatory logic. A proof that extensionality in $\nm{SKI}$
  combinator calculus follows from five axioms is given in \cite{hindley-seldin},
  theorem 8.14. Our result is similar, the key differences being that on the one hand we also
  incorporate linear and affine logic, while on the other hand our theorem is restricted to
  simply-typed lambda calculus.
\end{remarks}

% TODO: other cases of dependent functors

\subsection{Categorical prerelations}

Note that the axioms $\rightId,$ $\leftId,$ and $\compAssoc$ give every universe with (at least)
linear logic the shape of a weak category, which we can define as follows.

\begin{definition}
  For a layer 2 universe $\U$ with functors, a $\U$-valued reflexive and transitive prerelation
  $(\prec)$ on a set $S$ is defined to be \defn{categorical} if we have the following three
  instances for $a,b,c,d \in S.$
  \begin{itemize}
    \item $\begin{aligned}[t]
      \rightId_{ab} &: \prod_{f : a \prec b} [f \circ \id_a \equiv f]\\
                    &\ \ \eqDef [- \circ \id_a \equiv \idFun_{a{\prec}b}],
    \end{aligned}$
    \item $\begin{aligned}[t]
      \leftId_{ab} &: \prod_{f : a \prec b} [\id_b \circ f \equiv f]\\
                   &\ \ \eqDef [\id_b \circ - \equiv \idFun_{a{\prec}b}],
    \end{aligned}$
    \item $\begin{aligned}[t]
      \assoc_{abcd} &: \prod_{f : a \prec b, g : b \prec c, h : c \prec d} [(h \circ g) \circ f \equiv h \circ (g \circ f)]\\
                    &\ \ \eqDef [\compFun_{[b{\prec}c][[c{\prec}d]{\to}[b{\prec}d]][[c{\prec}d]{\to}[a{\prec}d]]}(- \circ -) \circ \revCompFun_{[c{\prec}d][b{\prec}d][a{\prec}d]} \circ (- \circ -) \equiv\\
                    &\ \ \quad\ \ \,\revCompFun_{[b{\prec}c][a{\prec}c][[c{\prec}d]{\to}[a{\prec}d]]}(- \circ -) \circ (- \circ -)].
    \end{aligned}$
  \end{itemize}
\end{definition}

In $U = \Set,$ these axioms obviously reduce to the axioms of a 1-category. However, since the
expanded versions have roughly the shape of triangle and pentagon identities, we conjecture that
$\Cat$-valued prerelations are precisely the same as bicategories.

If this conjecture holds, then theorem \ref{thm:layer2-extensionality} lets us mechanically
translate proofs written in the language of 1-categories (with equality replaced with instance
equivalence) to proofs about bicategories (or higher).

\subsection{Rewriting}

% TODO: If $p_a : \V$ is a type depending on $a : A$ in a suitable way, then for $a,b : A$ we have
% a functor $f_{ab} : a \equiv b \to [p_a \to p_b].$

\subsection{Singletons}

\subsection{Products and coproducts}

\subsection{Equivalences}

\subsection{Functor universe}

% TODO: adapt

%\begin{proposition}
%  \label{prp:fun-univ-full}
%  In this case, $\U^A$ also has internal functors, defined by $[B^A \to C^A] := [B \to C]^A$
%  for $B,C \in \U.$
%\end{proposition}
%\vspace{-1ex}
%\begin{proof}
%  For a functor $G : B^A \to C^A,$ which is also an instance of the type $[A \to B \to C],$
%  and an instance $F : B^A,$ which is also an instance of $[A \to B],$ we define their functor
%  application in $\U^A$ by $G(F) := \revSubstFun_{ABC}(G,F).$ This definition
%  respects instance equivalences because it is an application of the functor
%  $\revSubstFun_{ABC}(G).$
%\end{proof}
%
%\begin{proposition}
%  \label{prp:fun-univ-embed-map}
%  $\U$ embeds into $\U^A$: For each type $B \in \U$ and instance $b : B,$ we have
%  a corresponding instance $b^A := \constFun_{AB}(b) : B^A.$ This embedding
%  respects instance equivalences and functor application, up to instance equivalence.
%\end{proposition}
%\vspace{-1ex}
%\begin{proof}
%  As a functor, $\constFun_{AB}$ respects instance equivalences.
%  Moreover, for $G : B \to C$ and $b : B$ with $B,C \in \U$ we have an equivalence
%  \[\nm{embedMap}(G,b) : G^A(b^A) \eqDef \revSubstFun_{ABC}(\constFun_{A[B{\to}C]}(G),\constFun_{AB}(b)) \equiv
%    \constFun_{AC}(G(b)) \eqDef (G(b))^A\]
%  by lemma \ref{lem:functoriality-step-equiv}.
%\end{proof}
%
%\begin{proposition}
%  \label{prp:fun-univ-full-funop}
%  $\U^A$ has full functor operations, defined by $\idFun_{B^A} := (\idFun_B)^A,$
%  $\revAppFun_{(B^A)(C^A)} := (\revAppFun_{BC})^A, \dots$ for $B,C \in \U.$
%\end{proposition}
%\vspace{-1ex}
%\begin{proof}
%  Clearly these instances are functors of the correct type, and by $\nm{embedMap}$
%  they map embedded instances of types in $\U$ to the correct values. However, we
%  need to verify that they also map all other instances of $\U^A$ to the values
%  specified by their definition, up to equivalence. E.g.\ for $\idFun_{B^A}$ we
%  need to provide an equivalence
%  \[\fndef{\idFun_{B^A}}(F) : \idFun_{B^A}(F) \eqDef \revSubstFun_{ABC}(\constFun_{AB}(\idFun_B),F) \equiv F\]
%  for $F : B^A,$ and likewise for the other four combinators.
%
%  These equivalences can be obtained either via the algorithm given in theorem
%  \ref{thm:extensionality}, or directly from the lemmas leading up to that theorem.
%
%  Instance equivalences $\nm{rightId}, \nm{leftId}, \dots$ in $\U^A$ are trivially
%  obtained from the corresponding equivalences in $\U$ by repeated application of
%  $\nm{embedMap}.$
%\end{proof}
%
%\begin{proposition}
%  \label{prp:fun-univ-lift-fun}
%  If $G : B \to C$ ($B,C \in \U$) is a functor with definition
%  \begin{align*}
%    G : B &\to     C\\
%        b &\mapsto t_b
%  \end{align*}
%  where $t_b$ follows the constraints in proposition \ref{prp:functoriality},
%  then we can lift the family $(t_b)_{b : B}$ to a family $(T_F)_{F : B^A}$ of
%  instances of $C^A,$ which gives a definition
%  \begin{align*}
%    G^A : B^A &\to     C^A\\
%          F   &\mapsto T_F
%  \end{align*}
%  for the embedded functor $G^A.$
%\end{proposition}
%
%\begin{proof}
%  We recursively define $T_F$ based on the three possibilities for $t_b.$
%  \begin{itemize}
%    \item If $t_b$ is a constant $c$ independent of $b,$ we set $T_F := c^A.$
%    \item If $t_b = b,$ we set $T_F := F.$
%    \item If $t_b$ is a functor application, we set $T_F$ to the application of
%    the lifted functor to the lifted term.
%  \end{itemize}
%  Then, executing the functoriality algorithm for $(t_b)_{b : B}$ and
%  $(T_F)_{F : B^A}$ gives results $G' : B \to C$ and $G'' : B^A \to C^A$ that
%  only consist of functor applications of constant terms, such that each term
%  $x$ in $G'$ corresponds exactly to $x^A$ in $G''.$ Thus by repeated application
%  of $\nm{embedMap}$ we obtain an equivalence between $G'^A$ and $G''.$ Composing
%  this equivalence with the definitions of $G,$ $G',$ and $G''$ yields the
%  required equivalence $\fndef{G^A}(F) : G^A(F) \equiv T_F$ for $F : B^A.$
%\end{proof}
%
%We can use this proposition to obtain an alternative proof of proposition
%\ref{prp:functoriality} as follows. Given a family $(t_a)_{a : A}$ of terms
%$t_a : B$ that satisfy the constraints of proposition \ref{prp:functoriality},
%instead of executing the functoriality algorithm we may lift $(t_a)$ to a
%family $(T_G)_{G : A^A}$ of terms $T_G : B^A,$ and set the result to
%$T_{\idFun_A}.$
%
%This functor is equivalent to the functor $F$ constructed in the proof of
%proposition \ref{prp:functoriality} because according to proposition
%\ref{prp:fun-univ-lift-fun}, $F^A(G)$ is equivalent to $T_G$ for each $G : A^A,$
%and due to the following
%\begin{proposition}
%  For every $B \in \U$ and $F : A \to B$ we have an equivalence
%  \[\nm{embedId}(F) : F^A(\idFun_A) \equiv F.\]
%\end{proposition}
%\vspace{-2ex}
%\begin{proof}
%  By $\nm{rightId}.$
%\end{proof}
%
%\begin{remark}
%  This shows that the functor universe $\U^A$ can be regarded as $\U$ with an
%  adjoined element $\idFun_A : A^A,$ which behaves similarly to the symbol $X$
%  in a polynomial ring $R[X].$
%\end{remark}
%
%\begin{proposition}
%  \label{prp:fun-univ-lift-equiv}
%  Let $B,C \in \U,$ $(s_b)_{b : B}$ and $(t_b)_{b : B}$ be families of instances
%  $s_b,t_b : C,$ and $(e_b)_{b : B}$ be a family of instance equivalences
%  $e_b : s_b \equiv t_b$ that satisfy the constraints of theorem
%  \ref{thm:extensionality}.
%
%  Then $(e_b)_{b : B}$ lifts to a family $(E_F)_{F : B^A}$ of equivalences
%  $E_F : S_F \equiv T_F,$ where $(S_F)$ and $(T_F)$ are lifted from $(s_b)$
%  and $(t_b)$ according to proposition \ref{prp:fun-univ-lift-fun}.
%\end{proposition}
%
%\begin{proof}
%  We recursively define $E_F$ based on the six possibilities for $e_b.$
%  \begin{itemize}
%    \item If $e_b$ is a constant $e : c \equiv c',$ set
%    $E_F := \constFun_{AC}(e) : c^A \equiv c'^A.$
%    \item If $e_b = \id_{t_b},$ set $E_F := \id_{T_F}.$
%    \item If $e_b$ is $f_b^{-1}$ obtain $E_F$ by recursion.
%    \item Likewise for $e_b = g_b \circ f_b.$
%    \item If $e_b = G_b(f_b),$ set $E_F := I_F(H_F),$ where $H_F$ is obtained recursively
%    from $f_b,$ and $I_F$ is lifted from $G_b.$
%    \item If $e_b = \fndef{G}(x_b,x'_b,\dots)$ for a combinator $G,$ set
%    $E_F := \fndef{G^A}(X_F,X'_F,\dots),$ where $X_F$ is lifted from $x_b,$ etc.
%    \qedhere
%  \end{itemize}
%\end{proof}
%
%This gives rise to an alternative proof of theorem \ref{thm:extensionality} as follows.
%Given a family $(e_a)_{a : A}$ of equivalences $e_a : s_a \equiv t_a$ as in theorem
%\ref{thm:extensionality}, lift $(e_a)$ to a family $(E_H)_{H : A^A}$ of equivalences
%$E_H : S_H \equiv T_H$ in the functor universe $\U^A.$ Since $S_{\idFun_A}$ and
%$T_{\idFun_A}$ are equivalent to the functors that the functoriality algorithm
%produces for $(s_a)$ and $(t_a),$ respectively, the result can be obtained from
%$E_{\idFun_A}.$







\subsection{Universe type}
\label{sec:layer2-universe-type}

%TODO: Define $\V$ as type in $\U.$
%
%\begin{definition}
%  If $\U$ has functors, a \defn{property} $\phi$ on a type $A : \U$ is defined a functor
%  $\phi : A \to \V.$ A \defn{relation} $\psi$ on $A : \U$ is a functor $\psi : A \to A \to \V$
%  (or equivalently $\psi : A \times A \to \V$ if $\U$ has products).
%  % TODO: Apparently, for categories we need to replace one A with A^op. Or is that already the
%  % case for the definition of the product type? Or maybe morphisms are just a prerelation, and
%  % isomorphisms are a relation?
%\end{definition}
%
%Properties and relations in $\Unit$ and $\Prop$ are trivial. A property on $A : \Set$ is equivalent
%to a subset of $A.$ A relation on $A : \Set$ is equivalent to a set-theoretic relation. In $\Cat,$
%morphisms and isomorphisms form relations.
%
%\subsection{Dependent functors}
%
%\begin{definition}
%  We say that a layer 2 universe $(\U,\V)$ has \defn{dependent functors} if it has functors and for
%  every type $A : \U$ and property $\phi$ on $A$ there is a \defn{dependent functor type}
%  $[\Pi\,a : A,\:\phi(a)] : \V$ along with a dependent function $\apply_{\phi}$ that maps pairs of
%  instances $p : \Pi\,a : A,\:\phi(a)$ and $a : A$ to a value $\apply_{\phi}(p,a) : \phi(a).$
%\end{definition}
%
%Note that the notation is chosen such that ``$\phi(a)$'' in ``$[\Pi\,a : A,\:\phi(a)]$'' can be
%replaced by an expression that denotes a functor description.
%
%$\Unit$ and $\Bool$ trivially (and uninterestingly) have dependent functors. For $A : \Set,$ we take
%\[[\Pi\,a : A,\:\phi(a)] := \begin{cases}
%  \top & \text{if } \phi(a) = \top \text{ for all } a \in A,\\
%  \bot & \text{otherwise}
%\end{cases}\]
%and let $\apply$ map everything to the single instance $\star : \top.$
%
%In the universe $\Cat,$ we can \emph{not} define dependent functors as types of $\V = \Set$ because
%$A$ can be $\Set$ as well, with the consequence that the collection of instances of
%$[\Pi\,a : A,\:\phi(a)]$ would be too large to be a set. Therefore, in section \ref{sec:layer2-} we
%will describe algorithms to convert proofs that use dependent functors to proofs that use weaker
%structure.
%
%TODO: We need functorial type constructors here, including $\Pi.$ But without instance equivalences,
%we can't specify what they map to. Should layer 2- come before layer 2 after all?
%
%\begin{definition}
%  A layer 2 universe $(\U,\V)$ with dependent functors has \defn{dependent logic} if we have
%  the following three instances of dependent functors for $A,B : \U.$
%  \begin{itemize}
%    \item $\constPi_{A} : \Pi\,p : \V,\:[p \to [\Pi\,a : A,\:p]]$
%    \item $\appPi_{A} : \Pi\,\phi : A \to \V,\:\Pi\,\psi : A \to \V,\:[[\Pi\,a : A,\:[\phi(a) \to \psi(a)]] \to [\Pi\,a : A,\:\phi(a)] \to [\Pi\,a : A,\:\psi(a)]]$
%    \item $\compFunPi_{AB} : \Pi\,F : A \to B,\:\Pi\,\phi : B \to \V,\:[[\Pi\,b : B,\:\phi(b)] \to [\Pi\,a : A,\:\phi(F(a))]]$
%  \end{itemize}
%\end{definition}
%
%TODO: Dependent functor description
%
%\begin{proposition}
%  \label{prp:layer2-dependent-functoriality}
%  Let $(\U,\V)$ be a universe with dependent logic, $A : U,$ $\phi$ be a property on $A,$
%  and $(t_a)_{a : A}$ be a family of instances $t_a : \phi(a)$ that are one of the following:
%  \begin{itemize}
%    \item a constant independent of $a$ (which implies that $\phi$ is constant),
%    \item a functor application $p_a(q_a)$ with $p_a : \psi(a) \to \phi(a)$ and $q_a : \psi(a)$ for
%    a property $\psi$ on $A,$ such that $p_a$ and $q_a$ recursively follow the same rules, or
%    \item a dependent functor application $p_a(b_a)$ with $p_a : \Pi\,b : B,\:\psi(b)$ and $b_a : B$
%    for a type $B : \U$ and property $\psi$ on $B,$ such that $p_a$ recursively follows the same
%    rules and $b_a$ matches the restrictions of proposition \ref{prp:layer1-functoriality}.
%  \end{itemize}
%  Then we can construct a dependent functor
%  \begin{align*}
%    p : A &\to     \phi(a)\\
%        a &\mapsto t_a.
%  \end{align*}
%\end{proposition}

\section{Layer 3}

\bibliographystyle{plain}
\bibliography{UniverseAbstractions}

\end{document}
